% !TeX root = ./document.tex
\documentclass[document]{subfiles}
\begin{document}
\chapter{Метрические пространства}
Начнём с того, что все знают, надо ведь с чего-то начать. Мы будем несколько раз возвращаться к метрическим пространствам, а не изучим всё сразу. Один из полезных результатов, который мы получим, этоновое описание компакта в метрических пространствах. Он будет
самым рабочим. А компакт~--- вещь очень полезная. Компакты в гигантских пространствах напоминают компакты в $\bR^n$ или в $\bC^n$ и обладают теми же полезными свойствами.

\begin{definition}[Метрика]
    $X\text{~--- множество}, \rho: X \times X \rightarrow \bR$

    $\rho$~--- \textbf{метрика} $\Leftrightarrow$
    \begin{enumerate}
        \item $\forall x \in X \forall y \in X (\rho(x,y) \geq 0 \land (\rho(x,y) = 0 \iff x = y))$
        \item $\forall x \in X \forall y \in X \rho(y,x) = \rho(x,y)$
        \item $\forall x \in X \forall y \in X \forall z \in X \rho(x,z) \leq \rho(x,y) + \rho(y,z)$
    \end{enumerate}
\end{definition}

Введём стандартное обозначение открытого шара. $x \in X, r > 0$

$B_r(x) = \{y \in X | \rho(x,y) < r \}$~--- шар с радиусом $r$.
$ \{B_r(x) \}_{r > 0}$~--- база окрестности в точке $x$.

$G$~--- открытое $\Leftrightarrow \forall x \in G \, \exists r > 0 B_r(x) \subseteq G$.

$F \text{~--- замкнутое} \Leftrightarrow F \subseteq X \land X \setminus F\text{~--- открытое}$.

В метрическом пространстве удобно характеризовать замкнутое множества с помощью последовательностей. Вспомним, что такое сходящаяся последовательность.

\begin{definition}
$\{x_n\}_{n=1}^\infty\text{~--- последовательность в } X$

\[\liml_{n \to \infty} x_n = x_0 \Leftrightarrow \liml_{n \to \infty} \rho(x_n, x_0) = 0\]
\end{definition}

\begin{theorem}
$(X, \rho)\text{~--- метрическое пространство}$

$F$~--- замкнутое $\Leftrightarrow$ $\{x_n\}_{n=1}^\infty$~--- последовательность в $F \implies (\liml_{n \to \infty } x_n = x_0 \implies x_0 \in F)$
\end{theorem}


\begin{definition}[Фундаментальная последовательность]
    $\{x_n\}^\infty_{n=1} \text{~--- фундаментальная} \Leftrightarrow \forall \varepsilon > 0 \exists N \in \bN \forall n \in \bN \forall m \in \bN ((n > N \land m > N) \implies \rho(x_n, x_m) < \varepsilon) \Leftrightarrow \liml_{n,m \to \infty} \rho(x_n,x_m) = 0 $
\end{definition}


\begin{remark}
    $\exists x_0 \liml_{n \to \infty} x_n = x_0 \Rightarrow \{x_n\}^\infty_{n=1}\text{~--- фундаментальная}$
\end{remark}


\begin{definition}[Полное метрическое пространство]
    $(X, \rho)$~--- полное $\Leftrightarrow$ все фундаментальные последовательности имеют предел, лежащий в $X$
\end{definition}

Почему хорошо жить в полном метрическом пространстве?

\begin{remark}[о пользе полноты]
    $F: X \rightarrow \bR, (X,\rho)$~--- метрическое пространство, $F$~--- непрерывная.

    Стоит задача найти $x_0 \in X$ т.ч. $F(x_0) = 0$

    Положим, есть алгоритм, который порождает последовательность $\seq{x_n}^\infty_{n=1}, \liml_{n \to \infty} F(x_n) = 0, \liml_{n, m \to \infty} \rho(x_n, x_m) = 0$.

    Тогда если $(X, \rho)$~--- полное, то $\liml_{n \to \infty} x_n = x_0, F(x_0) = 0$.

    А если нет, то из наших вычислений вообще ничего не следует, возможно, решения вообще нет.
\end{remark}

\begin{example}
    $\bR^n, \bC^n$~--- полные.
\end{example}

\begin{example}
    $\bR^n \setminus \{\mathbb{0}_n\}$~--- неполное.
\end{example}

\begin{example}
    $\bQ$~--- неполное.
\end{example}

Потом приведем примеры поинтереснее. Кстати, древние греки пришли в ужас, когда узнали, что $\bQ$~--- неполное.

\begin{definition}[ограниченное множество]
    $(X,\rho)\text{~--- метрическое пространство}, A \subseteq X, A$~--- ограниченное, если 
    \[ \exists R > 0 \exists x_0 \in X A \subseteq B_R(x_0) \]
\end{definition}

\begin{theorem}[Свойства фундаментальных последовательностей]
\label{theo:cauchy-seq-properties}
    $(X,\rho)$~--- метрическое пространство, $\{x_n\}^\infty_{n=1}$~--- фундаментальная последовательность $\Rightarrow$
    \begin{enumerate}
        \item $\{x_n\}^\infty_{n=1}$~--- ограниченная, т.е. $\exists R > 0 \, \exists x_0 \in X \, \forall n \in \bN \, x_n \in B_R(x_0)$
        \item $\{x_{n_k}\}^\infty_{k=1}\text{~--- подпоследовательность } \Rightarrow (\exists a \in X \liml_{k \to \infty} x_{n_k} = a \implies \exists a \in X \liml_{n \to \infty} x_n = a = \liml_{k \to \infty} x_{n_k}) $
        \item $\{ \varepsilon_k \}_{k=1}^\infty$~--- произвольная последовательность в $\bR, \forall k \in \bN \varepsilon_k > 0 \Rightarrow \exists \{x_{n_k}\}^\infty_{k=1}$~--- подпоследовательность $\forall j \in \bN (j > k \implies \rho(x_{n_k}, x_{n_j}) < \varepsilon_k)$
    \end{enumerate}
\end{theorem}

\begin{proof}[1 утверждение]
    Возьмём $\varepsilon = 1$, тогда из фундаментальности $\exists N \in \bN \forall n \in \bN (n > N \implies \rho (x_n, x_N) < 1)$.

    Возьмём $R = \max \{ \rho(x_1, x_N), \ldots, \rho(x_{N-1},x_{N}) \} + 1$.
        
    Тогда $\forall n \in \bN x_n \in B_R(x_N)$.
\end{proof}
\begin{proof}[2 утверждение]
    Возьмём $\varepsilon > 0$, тогда по фундаментальности $\exists N \in \bN \forall n \in \bN \forall m \in \bN ((\textcolor{blue}{\underline{n > N}} \land  m > N) $ $\implies \underline{\rho(x_n, x_m)} < \varepsilon)$. Возьмём это $N$.

    $\exists a \liml x_{n_k} = a \Rightarrow \: \exists n_k (\rho(x_{n_k}, a) < \varepsilon \land \textcolor{blue}{\underline{n_k > N}})$. Возьмём это $n_k$.

    Возьмём некоторое $m > N$. Тогда $\rho(x_m, a) < \underline{\rho(x_m, x_{n_k})} + \rho(x_{n_k}, a) < 2 \varepsilon$
\end{proof}
\begin{proof}[3 утверждение]
    Докажем по индукции:

    $\varepsilon_1: \exists n_1 \forall n \in \bN \forall m \in \bN ((n > n_1 \land m > n) \implies \rho(x_m,x_n) < \varepsilon_1)$. Выберем $n_1$, тогда $\forall m \in \bN (m > n_1 \implies \rho(x_m, x_{n_1}) < \varepsilon_1)$.

    $\varepsilon_k:$ по индукции выбрали $n_1, \ldots, n_{k-1}$, $k \geq 2$. $\forall j \in ( 1 \ldots k-1 ) \forall m \in \bN (m > n_j \implies \rho (x_m, x_{n_j}) < \varepsilon_j)$.
    Из фундаментальности исходной последовательности $\exists n_k (n_k > n_{k-1} \land \forall m \in \bN (m > n_k \implies \rho (x_m, x_{n_k}) < \varepsilon_k))$
\end{proof}

\begin{corollary}
\label{cor:subseq-metric-sum-converges}
    $(X, \rho)$~--- метрическое пространство, $\{x_n\}$~--- фундаментальная последовательность $\Rightarrow$
    \[ \exists \{x_{n_k} \} \text {~--- подпоследовательность} \sum_{k=1}^\infty \rho(x_{n_k}, x_{n_{k+1}}) < + \infty \]
\end{corollary}
\begin{proof}
    По 3 свойству при $\varepsilon_k = \frac{1}{2^k}$.
\end{proof}

\begin{theorem}[О замкнутом подмножестве]
\label{theo:closed-subset-complete-space}
    $(X,\rho)$~--- метрическое пространство $\Rightarrow$
    \begin{enumerate}
        \item $(X, \rho)$~--- полное, $Y \subseteq X$, $Y$~--- замкнутое $\Rightarrow (Y, \rho)$~--- полное
        \item $Y \subseteq X$, $(Y,\rho)$~--- полное $\Rightarrow$ $Y$~--- замкнутое
    \end{enumerate}
\end{theorem}

\begin{proof}[1 утверждение]
        Доказательство следует прямо из определения. Знаем, что $Y$~--- замкнутое подмножество полного пространства.

        Берём $\seq{x_n}^\infty_{n=1}$~--- фундаментальную последовательность в Y.

        $\forall n \in \bN x_n \in X, X$~--- полное $\Rightarrow \exists x_0 \in X \liml_{n \to \infty} x_n = x_0$. Возьмём этот $x_0$.

        $Y$~--- замкнутое $\Rightarrow x_0 \in Y \Rightarrow (Y, \rho)$~--- полное.
\end{proof}
\begin{proof}[2 утверждение]
        Второй пункт не труднее первого. Пусть $\{x_n\}^\infty_{n=1}$~--- произвольная фундаментальная последовательность в $Y$.

        $(Y, \rho)$~--- полное $\Rightarrow \exists x_0 \in Y \liml_{n \to \infty} x_n = x_0 \Rightarrow Y$~--- замкнутое из-за произвольности последовательности.
\end{proof}

\section{Банаховы пространства}

Сначала введём понятие полунормы.
\begin{definition}[полунорма]
    $X$~--- линейное пространство над $\bC$ (или $\bR$).

    $p: X \rightarrow \bR$~--- полунорма $\Leftrightarrow$
    \begin{enumerate}
        \item $\forall x \in X \forall y \in X p(x + y) \leq p(x) + p(y)$ (полуаддитивность)
        \item $\forall x \in X \forall \lambda \in \bC p(\lambda x) = |\lambda| p(x)$
    \end{enumerate} 
\end{definition}

\begin{property}
    $p$~--- полунорма $\Rightarrow$
    \[ p(\mathbb{0}) = 0 \land \forall x \in X p(x) \geq 0 \]
\end{property}

\begin{proof}
    $p(\mathbb{0}) = p(0 \cdot \mathbb{0}) = 0 \cdot p(\mathbb{0}) = 0$.

    $\forall x \in X \mathbb{0} = x + (-x) \Rightarrow (x \in X \implies 0 = p(\mathbb{0}) = p(x + (-x)) \leq p(x) + \underbrace{p(-x)}_{p(x)} = 2 p(x)) \Rightarrow \forall x \in X p(x) \geq 0$
\end{proof}

\begin{definition}[Норма]
    $X$~--- линейное пространство, $p : X \rightarrow \bR$

    $p$~--- норма $\Leftrightarrow $ ($p$~--- полунорма $\land \forall x \in X (p(x) = 0 \iff x = \mathbb{0}$)).

    Будем обозначать $||x|| \coloneqq p(x)$.
\end{definition}
$(X, || \cdot ||)$ будем обозначать нормированное пространство и $x \in X, y \in X \Rightarrow \rho(x,y) \coloneqq ||x-y||$. Тогда $(X,||\cdot ||)$~--- метрическое пространство (аксиомы для метрики следуют прямо из определений).

\begin{definition}[банахово пространство]
    $(X, || \cdot ||)$~--- банахово, если оно полное
\end{definition}
Еще пару определений перед критерием банахова пространства.

\begin{definition}[подпространство в алгебраическом смысле]
    $X$~--- линейное пространство$,L \subseteq X, K \in \{ \bR, \bC \}$

    $L$~--- подпространство в алгебраическом смысле $\Leftrightarrow$ $\forall x \in L \forall y \in L \forall \alpha \in K \forall \beta \in K \alpha x + \beta y \in L$.
\end{definition}

\begin{definition}[подпространство]
    $(X, || \cdot ||)$, $L \subseteq X$, $L$~--- подпространство, если
    \begin{enumerate}
        \item $L$ подпространство в алгебраическом смысле
        \item $L = \overline{L}$ ($\overline{L}$~--- замыкание)
    \end{enumerate}
\end{definition}

Теперь нам потребуется сходимость рядов. Для того, чтобы говорить о сходимости, нужна топология.

\begin{definition}[Cходимость]
    $(X, || \cdot ||)$~--- метрическое пространство, $\seq{x_k}^\infty_{k=1}$~--- последовательность элементов $X$, $S_n = \sum^n_{k=1} x_k $

    $\sum_{k=1}^\infty x_k (*)$, $(*)$ сходится, если $\exists S \in X \liml_{n \to \infty}{S_n} = S$

    $(*)$ cходится абсолютно, если $\sum^\infty_{k=1} ||x_k||$ сходится
\end{definition}
В $\bR^n$ (или в $\bC^n$) если у нас была абсолютная сходимость, то была и обычная, но вообще говоря, это не так.

\begin{theorem}[Критерий полноты нормированного пространства]
    $(X, || \cdot||)$ - полное $\Leftrightarrow$ из абсолютной сходимости ряда следует сходимость ряда.
\end{theorem}

\begin{proof}
    Предположим, что наше \textcolor{blue}{пространство полное} ($\Rightarrow)$. $(X, \rho)$~--- полное, $\seq{x_k}^\infty_{k=1}$~--- последовательность, при этом
    \[ \sum^\infty_{k=1} ||x_k|| \tag{**} \text { сходится } \]
    Цель такая: последовательность $S_n$~--- фундаментальная. Сейчас применим критерий Коши к ряду (**). Это ряд из чисел, так что всё в порядке.
    Пусть $\varepsilon > 0$. По критерию Коши $\exists N \in \bN \forall n \in \bN \forall p \in \bN (n > N \implies \sum^{n+p}_{k=n+1} ||x_k|| < \varepsilon)$. Далее $N$, $n$, $p$ взяты отсюда, $n > N$.

    $S_n = \sum^n_{k=1} x_k$.
    \begin{multline*}
        ||S_{n+p} - S_n|| = \left|\left| \sum^p_{k=1}  x_{n+k}\right|\right| \underbrace{\leq}_{\text{полуаддитивность}} \sum^p_{k=1}||x_{n+k}|| = \sum^{n+p}_{k=n+1} ||x_k|| < \varepsilon\\
        \Rightarrow \seq{S_n}^\infty_{n=1} \text{~--- фундаментальная}\\
        \textcolor{blue}{\Rightarrow} \exists S \in X \liml_{n \to \infty}{S_n} = S \\
        \Rightarrow \sum^\infty_{k=1} x_k \text{ сходится }
    \end{multline*}

    Теперь \textcolor{red}{($\Leftarrow$)}. У нас кроме определения ничего нет. Возьмём какую-то фундаментальную последовательность. Откуда взять предел? Есть соотношения между элементами последовательности. Возьмём подпоследовательность, ведь у нас есть следствие \ref{cor:subseq-metric-sum-converges}! 
    Из свойств фундаментальных последовательностей, мы знаем, что
    \begin{gather*}
        \exists \{x_{n_k}\}_{k=1}^\infty\text{~--- подпоследовательность} ||x_{n_1}|| + \sum_{k=1}^\infty ||x_{n_{k+1}} - x_{n_k}|| \text{ сходится} \\
         \textcolor{red}{\Rightarrow} \text{последовательность } x_{n_1} + \sum^{\infty}_{k=1} (x_{n_{k+1}} - x_{n_k}) \text{ сходится}
    \end{gather*}
    Но её последовательность частичных сумм~--- это в точности оригинальная подпоследовательность:
    \begin{gather*}
        S_m = x_{n_1} + \sum^{m-1}_{k=1}(x_{n_{k+1}} - x_{n_k}) = x_{n_m} \Rightarrow \exists S \in X \liml_{k \to \infty} x_{n_k} = S
    \end{gather*}
    Далее из части 2 Теоремы \ref{theo:cauchy-seq-properties}
    \[ \exists S \in X \liml_{k \to \infty} x_{n_k} = S \Rightarrow \exists S \in X \liml_{n \to \infty} x_{n} = S \]
\end{proof}

\section{Пространства ограниченных функций}
\begin{definition}
    Пусть $A$~--- произвольное множество. Стандартное обозначение $m(A)$~--- множество всех ограниченных функций из него в комплексные (или только в действительные, не важно) числа
    \[ m(A) = \{ f | f: A \rightarrow \bC \land \sup_{x \in A} |f(x)| < +\infty \} \]
\end{definition}

\begin{definition}
$f \in m(A) \Rightarrow$

\[||f||_{\infty} \coloneqq \sup_{x \in A} |f(x)|\]
\end{definition}

Это метрика для $m(A)$? Да, сейчас получим даже более сильный результат.

\begin{theorem}
\label{theo:bounded-func-space-is-banach}
    $(m(A), || \cdot ||_{\infty})$~--- банахово пространство
\end{theorem}

\begin{proof}
    Нужно проверить две вещи. Во-первых, что $|| \cdot ||_\infty$ удовлетворяет аксиомам нормы. А во-вторых, что пространство с таким определением является полным.
    Просто по определению, никаких хитрых критериев~--- возьмём фундаментальную последовательность и покажем, что у неё есть предел.

    Проверяем, что $|| \cdot ||_\infty$ удовлетворяет аксиомам нормы.

    \[ \forall \lambda \in \bC ||\lambda f||_\infty = \sup_{x \in A} (|\lambda| \cdot ||f(x)||) = |\lambda| \cdot \sup_{x \in A} ||f(x)|| = |\lambda| \cdot ||f||_\infty \]

    Нужно проверить неравенство треугольника.

\begin{gather*}
    \forall f \in m(A) \forall g \in m(A) \forall x \in A |f(x) + g(x)| \leq |f(x)| + |g(x)| \leq ||f||_\infty + ||g||_\infty \\
    \Rightarrow \forall f \in m(A) \forall g \in m(A) ||f+g||_\infty = \sup_{x \in A} |f(x) + g(x)| \leq ||f||_\infty + ||g||_\infty
\end{gather*}

    Следующая аксиома нормы:
    \[ ||f||_\infty = 0 \iff \forall x \in A f(x) = 0  \text{ т.е. $f$~--- нулевая функция} \]
    Теперь мы проверили аксиомы нормы. Доказываем полноту. 
    $\{f_n\}^\infty_{n=1}$~--- фундаментальная в $m(A)$.
\begin{multline*}
    \forall \varepsilon > 0 \exists N \in \bN \forall m \in \bN \forall n \in \bN(\\
    (m > N \land n > N) \implies ||f_n - f_m||_\infty < \varepsilon) \text{ т.е. } \sup_{x \in A} |f_n(x) - f_m(x)| < \varepsilon
\end{multline*}
    Первый вопрос: откуда взять претендента на роль предела? Еще желательно, чтобы он был единственный. Берём $\varepsilon, N, m, n$ из формулы выше, фиксируем $x$. Если для супремума есть неравенство, то и для $x$ тем более.
    $\forall x \in A ((n > N \land m > N) \implies |f_n(x) - f_m(x)| < \varepsilon) \Rightarrow \{f_n(x)\}^\infty_{n=0}$~--- фундаментальная последовательность чисел в $\bC$.
     
    $\Rightarrow \forall x \in A \: \exists L \in \bC \liml_{n \to \infty} f_n(x) = L$

    Определим $f \coloneqq (x \in A \mapsto \liml_{n \to \infty} f_n(x))$
    \begin{gather*}
        ((n > N \land m > N) \implies \forall x \in A |f_n(x) - f_m(x)| < \varepsilon) \quad \text { пусть } m \to \infty \\
        \Rightarrow (n > N \implies \forall x \in A |f_n(x) - f(x)| \leq \varepsilon) \\
        \Rightarrow (n > N \implies \underline{||f_n-f||_\infty} = \sup_{x \in A} |f_n(x) - f(x)| \underline{\leq \varepsilon})
    \end{gather*}
     Последнее сображение, которое нужно добавить, это то, что $f$~--- элемент $A$. Для $n > N$ можем записать $f$ как $f = (f - f_n) + f_n, f_n \in m(A)$. $\underline{||f_n - f||_\infty} = ||f - f_n||_\infty \Rightarrow f-f_n \in m(A)$.

     \[||f||_\infty = ||(f - f_n) + f_n||_\infty \leq ||f - f_n||_\infty + ||f_n||_\infty < +\infty \Rightarrow f \in m(A) \]
 \end{proof}
 Давайте заметим, что у нас получилось определение равномерной сходимости из математического анализа.

\[ \liml_{n \to \infty} f_n = f \in m(A) \Leftrightarrow \liml_{n \to \infty} \sup_{x \in A} |f_n(x) - f(x)| = 0 \Leftrightarrow f_n \darrow{\begin{subarray}{c}x \in A \\ n \to \infty\end{subarray}}  f \] 

\begin{definition}[Топологический компакт]
    Множество $K$~--- топологический компакт $\Leftrightarrow$
    \begin{enumerate}
        \item $((\forall \alpha \in A G_\alpha$~--- открытое множество $\land K \subseteq \bigcup_{\alpha \in A} G_\alpha) \implies \exists \{\alpha_j\}_{j=1}^n$~--- конечная последовательность элементов A $ K \subseteq \bigcup^n_{j=1} G_{\alpha_j})$ (т.е. у каждого покрытия K открытыми множествами существует конечное подпокрытие)
        \item Хаусдорфовость: $\forall x \in K \forall y \in K (x \ne y \implies \exists U \exists V (U$~---  открытое множество $\land V$~--- открытое множество $\land x \in U \land y \in V \land U \cap V = \varnothing))$
    \end{enumerate}
\end{definition}

\begin{definition}
    $C(K) = \{ f | f: K \rightarrow \bR \land f \text { непрерывна} \}$
    \[ ||f||_{C(K)} \coloneqq ||f||_\infty = \sup_{x \in K} |f(x)| \]
\end{definition}

\begin{corollary}
    $K$~--- топологический компакт $\Rightarrow (C(K), ||\cdot||_\infty)$~--- банахово
\end{corollary}

\begin{proof}
    Имеем дело с непрерывными функциями на компакте, поэтому \textcolor{orange}{$C(K) \subseteq m(K)$}. Напоминание: \textcolor{orange}{$m(K)$~--- полное}.

    Пусть $\seq{f_n}_{n=1}^\infty$~--- произвольная фундаментальная последовательность в $C(K)$, $f \in m(K)$~--- её предел (существует по теореме \ref{theo:bounded-func-space-is-banach}). Это значит, что $f_n(x) \darrow{\begin{subarray}{c}x \in K\\ n \to \infty\end{subarray}} f(x)$. Тогда из анализа знаем, что $f \in C(K)$. Из-за произвольности выбора последовательности \textcolor{orange}{$C(K)$ замкнуто}.

    \textcolor{orange}{Теорема \ref{theo:closed-subset-complete-space} $\Rightarrow$} $(C(K), ||\cdot||_\infty)$~--- полное $\Rightarrow (C(K), ||\cdot||_\infty)$~--- банахово.
\end{proof}


\section{Пространство последовательностей с $\sup$ нормой}

\begin{definition}
    $n \in \bN, l^\infty_n = \bC^n, x \in l^\infty_n$
    \[ ||x||_\infty \coloneqq \max_{j \in (1\ldots n)} |x_j| \]
    Это норма для $l^\infty_n$.
\end{definition}

Неформальное рассуждение: если вместо кортежей длины $n$ взять функции из $(1\ldots n)$ и переформулировать утверждения об элементах $l^\infty_n$ соответствующим образом, то в некотором смысле $l^\infty_n$ <<такое же>>, как пространство $m((1\ldots n))$. В частности, оно тоже полно.

Напомним, что последовательность~--- это функция на множестве натуральных чисел. Далее это используется явно.

\begin{definition}[$l^\infty$]
    $l^\infty \coloneqq m(\bN)$~--- множество пространства последовательностей, $||f||_\infty \coloneqq \sup_{j \in \bN} |f(j)|$~--- метрика на нём.
\end{definition}

$l^\infty$ является обобщением пространств $l^\infty_n$ на цепочки произвольных длин.

Отметим, что $l^\infty$~--- полное.

\begin{definition}
    \[ c \coloneqq \{ X \in l^\infty | \exists x_0 \in \bC  \liml_{n \to \infty}{X(n)} = x_0 \} \]
\end{definition}
\begin{definition}
    \[ c_0 \coloneqq \{X \in c | \liml_{n \to \infty} X(n) = 0\} \]
\end{definition}

$c, c_0$~--- замкнутые подпространства в $l^\infty \Rightarrow c, c_0$~--- банаховы. 

\section{Пространства $n$ раз непрерывно дифференцируемых функций на отрезке}

 \begin{definition}[норма $n$-ой производной]
    \[n \in \bN \quad C^{(n)}[a,b] = \{ f | f: [a, b] \rightarrow \bR \land \exists f^{(n)} \in C[a,b] \} \]
    \[ \norm{f}_{C^{(n)}[a,b]} = \max_{0 \leq k \leq n} \norm{f^{(k)}}_\infty, f^{(0)} = f \]
 \end{definition}

\begin{theorem}
     $(C^{(n)}[a,b], ||\cdot||_{C^{(n)}})$~--- банахово.
\end{theorem}

\begin{proof}
    \begin{gather*}
        \{f_m\}^\infty_{m=1} \text{~--- фундаментальная последовательность в } C^{(n)}[a,b] \\
        \textcolor{red}{\forall \varepsilon > 0 \exists N \in \bN \forall m \in \bN \forall q \in \bN ((m > N \land q > N) \implies} \normunder{f_m-f_q}{C^{(n)}} < \varepsilon) \\
        \Rightarrow \textcolor{red}{\ldots} \forall k \in (0\ldots n) \normunder{f_m^{(k)} - f_q^{(k)}}{\infty} < \varepsilon\\
        \Rightarrow \forall k \in (0\ldots n) \seq{f_m^{(k)}}_{m=1}^\infty \text{~--- фундаментальная}
    \end{gather*}

    \begin{multline*}
        \forall k \in (0\ldots n) \{f_m^{(k)} \} \text {~--- фундаментальная в полном пространстве } C[a,b]\\
        \Rightarrow \forall k \in (0\ldots n) \exists \varphi_k \in C[a,b] f_m^{(k)}(x) \darrow{\begin{subarray}{c}x \in [a,b] \\ m \to \infty\end{subarray}} \varphi_k(x)
    \end{multline*}
    Возьмём эти $\varphi_k$.

    Из анализа известно, что $\forall x \in [a, b] \forall k \in (1\ldots n) \varphi_{k}(x) = \varphi_{0}^{(k)}(x)$ (почленное дифференцирование). Причём все $\varphi_k$ непрерывны. То есть произвольно выбранная фундаментальная последовательность сходится к функции, непрерывно дифференцируемой $n$ раз.
 \end{proof}


\end{document}
