% !TeX root = ./document.tex
\documentclass[document]{subfiles}
\begin{document}
\chapter{Метрические пространства}
Начнём с того, что все знают. Надо ведь с чего-то начать. Мы будем несколько раз к ним возвращаться, а не изучим всё сразу. Один из полезных результатов~--- новое описание компакта в метрических пространствах. Он будет
самым рабочим. А компакт~--- вещь очень полезная. Компакты в гигантских пространствах напоминают компакты в $\bR^n$ или в $\bC^n$ и обладают теми же полезными свойствами.

\begin{definition}[Метрика]
    $X$~--- множество. $\rho: X \times X \rightarrow \bR$, $\rho$~--- \textbf{метрика}, если при $x \in X \land y \in X \land z \in X$ она обладает следующими свойствами 
    \begin{enumerate}
        \item $\rho(x,y) \geq 0 \land (\rho(x,y) = 0 \Leftrightarrow x = y)$
        \item $\rho(y,x) = \rho(x,y)$
        \item $\rho(x,z) \leq \rho(x,y) + \rho(y,z)$
    \end{enumerate}
\end{definition}

Введём стандартное обозначение открытого шара. $x \in X, r > 0$

$B_r(x) = \{y \in X: \rho(x,y) < r \}$~--- шар с радиусом $r$.
$ \{B_r(x) \}_{r > 0}$~--- база окрестности в точке $x$.

$G$~--- открытое, если $\forall x \in G \, \exists r > 0 B_r(x) \subset G$.

$F \text{~--- замкнутое} \Leftrightarrow F \subset X \land X \setminus F\text{~--- открытое}$.

В метрическом пространстве удобно характеризовать замкнутое множества с помощью последовательностей. Вспомним, что такое сходящаяся последовательность.

$\{x_n\}_{n=1}^\infty\text{~--- последовательность} \land \forall n \in \bN x_n \in X \land \liml_{n \to \infty} x_n = x_0 \Leftrightarrow \liml_{n \to \infty} \rho(x_n, x_0) = 0$

$(X, \rho)\text{~--- метрическое пространство} \Rightarrow (F$~--- замкнутое $\Leftrightarrow$ $\{x_n\}_{n=1}^\infty$~--- последовательность $\land \forall n \in \bN x_n \in F \land (\liml_{n \to \infty } x_n = x_0 \Rightarrow x_0 \in F))$


\begin{definition}[Фундаментальная последовательность]
    $\{x_n\}^\infty_{n=1} \text{~--- фундаментальная} \Leftrightarrow \forall \varepsilon > 0 \exists N \in \bN \forall n \in \bN \forall m \in \bN ((n > N \land m > N) \Rightarrow \rho(x_n, x_m) < \varepsilon) \Leftrightarrow \liml_{n,m \to \infty} \rho(x_n,x_m) = 0 $
\end{definition}


\begin{remark}
    $\exists x_0 \liml_{n \to \infty} x_n = x_0 \Rightarrow \{x_n\}^\infty_{n=1}\text{~--- фундаментальная}$
\end{remark}


\begin{definition}[Полное метрическое пространство]
    $(X, \rho)$~--- полное, если все фундаментальные последовательности имеют предел, лежащий в $X$
\end{definition}

Почему хорошо жить в полном метрическом пространстве?

\begin{remark}[о пользе полноты]
    $F: X \rightarrow \bR, (X,\rho)$~--- метрическое пространство, $F$~--- непрерывная.

    Стоит задача найти $x_0 \in X$ т.ч. $F(x_0) = 0$ \\
    Алгоритм: $\{x_n\}^\infty_{n=1}, \liml_{n \to \infty} F(x_n) = 0, \liml_{n, m \to \infty} \rho(x_n, x_m) = 0$
    Если $(X, \rho)$~--- полное, то $\liml_{n \to \infty} x_n = x_0, F(x_0) = 0$
    А если нет, то из наших вычислений вообще ничего не следует, возможно, решения вообще нет.
\end{remark}

\begin{example}
    $\bR^n, \bC^n$~--- полные.
\end{example}

\begin{example}
    $\bR^n \setminus \{\mathbb{0}_n\}$~--- неполное.
\end{example}

\begin{example}
    $\bQ$~--- неполное.
\end{example}

Потом приведем примеры поинтереснее. Кстати, древние греки пришли в ужас, когда узнали, что $\bQ$~--- неполное.

\begin{definition}
    $(X,\rho)\text{~--- метрическое пространство}, A \subset X, A$~--- ограниченное, если 
    \[ \exists R > 0 \exists x_0 \in X A \subset B_R(x_0) \]
\end{definition}

\begin{theorem}[Свойства фундаментальных последовательностей]
    $(X,\rho)$~--- метрическое пространство, $\{x_n\}^\infty_{n=1}$~--- фундаментальная последовательность, тогда выполняется:
    \begin{enumerate}
        \item $\{x_n\}^\infty_{n=1}$~--- ограниченная, т.е. $\exists R > 0 \, \exists x_0 \in X \, \forall n \in \bN \, x_n \in B_R(x_0)$
        \item $\exists \{x_{n_k}\}^\infty_{k=1}\text{~--- подпоследовательность } \exists a \liml x_{n_k} = a \Rightarrow \liml_{n \to \infty} x_n = a $
        \item $\{ \varepsilon_k \}_{k=1}^\infty$~--- произвольная последовательность действительных чисел $\land \forall k \in \bN \varepsilon_k > 0 \Rightarrow \exists \{x_{n_k}\}^\infty_{k=1}$~--- подпоследовательность $\forall j \in \bN (j > k \Rightarrow \rho(x_{n_k}, x_{n_j}) < \varepsilon_k)$
    \end{enumerate}
\end{theorem}

\begin{proof}[1 утверждение]
    Возьмём $\varepsilon = 1$, тогда из фундаментальности $\exists N \, \forall n \in \bN (n > N \Rightarrow \rho (x_n, x_N) < 1)$.

    Возьмём $R = \max \{ \rho(x_1, x_N), \ldots, \rho(x_{N-1},x_{N}) \} + 1$. Единичка на всякий случай.
        
    Тогда $\forall n \in \bN x_n \in B_R(x_N)$.
\end{proof}
\begin{proof}[2 утверждение]
    Возьмём $\varepsilon > 0$, тогда по фундаментальности $\exists N \forall n \in \bN \forall m \in \bN ((\textcolor{blue}{\underline{n > N}} \, \land \, m > N) $ $\Rightarrow \underline{\rho(x_n, x_m)} < \varepsilon)$. Возьмём это $N$.

    $\exists a \liml x_{n_k} = a \Rightarrow \exists n_k (\rho(x_{n_k}, a) < \varepsilon \land \textcolor{blue}{\underline{n_k > N}})$. Возьмём это $n_k$.

    Возьмём некоторое $m > N$. Тогда $\rho(x_m, a) < \underline{\rho(x_m, x_{n_k})} + \rho(x_{n_k}, a) < 2 \varepsilon$
\end{proof}
\begin{proof}[3 утверждение]
    Докажем по индукции:

    $\varepsilon_1: \exists n_1 \forall n \in \bN \forall m \in \bN ((n > n_1 \land m > n) \Rightarrow \rho(x_m,x_n) < \varepsilon_1)$. Выберем $n_1$, тогда $\forall m \in \bN (m > n_1 \Rightarrow \rho(x_m, x_{n_1}) < \varepsilon_1)$.

    $\varepsilon_k:$ по индукции выбрали $n_1, \ldots, n_{k-1}$, $k \geq 2$. $\forall j \in ( 1 \ldots k-1 ) \forall m \in \bN (m > n_j \Rightarrow \rho (x_m, x_{n_j}) < \varepsilon_j)$.
    Из фундаментальности исходной последовательности $\exists n_k (n_k > n_{k-1} \land \forall m \in \bN (m > n_k \Rightarrow \rho (x_m, x_{n_k}) < \varepsilon_k))$
\end{proof}

\begin{corollary}
\label{cor:subseq-metric-sum-converges}
    $(X, \rho)$, $\{x_n\}$~--- фундаментальная последовательность, тогда 
    \[ \exists \{x_{n_k} \} \text { т.ч. } \sum_{k=1}^\infty \rho(x_{n_k}, x_{n_{k+1}}) < + \infty \]
\end{corollary}
\begin{proof}
    По 3 свойству при $\varepsilon_k = \frac{1}{2^k}$.
\end{proof}

\begin{theorem}[О замкнутом подмножестве]
    $(X,\rho)$~--- метрическое пространство, тогда
    \begin{enumerate}
        \item $(X, \rho)$~--- полное, $Y \subseteq X$, $Y$~--- замкнутое $\Rightarrow (Y, \rho)$~--- полное
        \item $Y \subseteq X$, $(Y,\rho)$~--- полное $\Rightarrow$ $Y$~--- замкнутое
    \end{enumerate}
\end{theorem}

\begin{proof}[1 утверждение]
        Доказательство следует прямо из определения. Знаем, что $Y$~--- замкнутое поднмножество полного пространства.
        Берем фундаментальную последовательность. $Y \subset X$, пусть $\{x_n\}^\infty_{n=1}, \forall n \in \bN x_n \in Y$~--- фундаментальная.
        $\forall n \in \bN x_n \in X, X$~--- полное $\Rightarrow \exists x_0 \in X \liml_{n \to \infty} x_n = x_0$. $Y$~--- замкнутое, значит $x_0 \in Y \Rightarrow (Y, \rho)$~--- полное.
\end{proof}
\begin{proof}[2 утверждение]
        Второй пункт не труднее первого. Пусть $\{x_n\}^\infty_{n=1}$~--- произвольная фундаментальная последовательность в $Y$.

        $Y$~--- полное $\Rightarrow \exists x_0 \in Y \liml_{n \to \infty} x_n = x_0 \Rightarrow Y$~--- замкнутое из-за произвольности последовательности.
\end{proof}

\section{Банаховы пространства}

Сначала введём понятие полунормы.
\begin{definition}[полунорма]
    Пусть $X$~--- линейное пространство над $\bR$ или $\bC$. Отображение $p: X \rightarrow \bR$ называется полунормой, если при $x \in X \land y \in X \land (\lambda \in \bR \lor \lambda \in \bC)$
    \begin{enumerate}
        \item $p(x + y) \leq p(x) + p(y)$ (полуаддитивность)
        \item $p(\lambda x) = |\lambda| p(x)$
    \end{enumerate} 
\end{definition}

\begin{property}
    $p$~--- полунорма $\Rightarrow$
    \[ \forall x \in X p(x) \geq 0 \land p(\mathbb{0}) = 0 \]
\end{property}

\begin{proof}
    $p(\mathbb{0}) = p(0 \cdot \mathbb{0}) = 0 \cdot p(\mathbb{0}) = 0$.
    Пусть $x \in X \Rightarrow \mathbb{0} = x + (-x) \Rightarrow p(\mathbb{0}) \leq p(x) + \underbrace{p(-x)}_{p(x)} = 2p(x) \Rightarrow p(x) \geq 0$
\end{proof}

\begin{definition}[Норма]
    $X$~--- линейное пространство, $p : X \rightarrow \bR$. $p$~--- норма $\Leftrightarrow $ ($p$~--- полунорма $\land $ ($p(x) = 0 \Leftrightarrow x = \mathbb{0}$)).
    Будем обозначать $||x|| := p(x)$.
\end{definition}
$(X, || \cdot ||)$ будем обозначать нормированное пространство. и при $(x \in X \land y \in X)$ $\rho(x,y) := ||x-y||$. Тогда $(X,||\cdot ||)$~--- метрическое пространство.

\begin{definition}[банахово пространство]
    $(X, || \cdot ||)$~--- банахово, если оно полное
\end{definition}
Еще пару определений перед критерием банахова пространства.

\begin{definition}[подпространство в алгебраическом смысле]
    $X$~--- линейное пространство, $L \subset X$. $L$~--- подпространство в алгебраическом смысле $\Leftrightarrow$ $\forall x \in L \forall y \in L \forall \alpha \in K \forall \beta \in K \alpha x + \beta y \in L$.
\end{definition}

\begin{definition}[подпространство]
    $(X, || \cdot ||)$, $L \subset X$, $L$~--- подпространство, если
    \begin{itemize}
        \item $L$ подпространство в алгебраическом смысле
        \item $L = \overline{L}$ ($\overline{L}$~--- замыкание)
    \end{itemize}
\end{definition}

Теперь нам потребуется сходимость рядов. Для того, чтобы говорить о сходимости, нужна топология.

\begin{definition}[Cходимость]
    \[(X, || \cdot ||) \quad \seq{x_k}^\infty_{k=1} \quad S_n = \sum^n_{k=1} x_k \]
    $\sum_{k=1}^\infty x_k (*)$,  
    (*) сходится, если $\exists \liml_{n \to \infty}{S_n} = S \in X$

    (*) cходится абсолютно, если $\sum^\infty_{k=1} ||x_k||$ сходится
\end{definition}
В $\bR^n$ (или в $\bC^n$) если у нас была абсолютная сходимость, то была и обычная, но вообще говоря, это не так.

\begin{theorem}[Критерий полноты нормированного пространства (банаховости)]
    $(X, || \cdot||)$ - полное $\Leftrightarrow$ из абсолютной сходимости ряда следует сходимость ряда.
\end{theorem}

\begin{proof}
    Предположим, что наше пространство полное ($\Rightarrow)$. $(X, \rho)$~--- полное, $\seq{x_k}^\infty_{k=1}$.
    \[ \sum^\infty_{k=1} ||x_k|| \tag{**} \text { сходится } \]
    Цель такая: последовательность $S_n$~--- фундаментальная. Сейчас применим критерий Коши к ряду (**). Это ряд из чисел, так что всё в порядке.
    Пусть $\varepsilon > 0$. По критерию Коши $\exists N \in \bN \forall n \in \bN \forall p \in \bN (n > N \Rightarrow \sum^{n+p}_{k=n} ||x_k|| < \varepsilon)$.

    $S_n = \sum^n_{k=1} x_k$.
    \begin{multline*}
        \sum^{n+p}_{k=n+1} ||x_k|| = \underline{||S_{n+p} - S_n||} = \left|\left| \sum^p_{k=1}  x_{n+k}\right|\right| \leq \sum^p_{k=1}||x_{n+k}|| \underline{< \varepsilon}\\
        \Rightarrow \underline{\seq{S_n}^\infty_{n=1} \text{~--- фундаментальная}}, (X, \rho) \text{~--- полное} \\
        \Rightarrow \exists S \in X \liml_{n \to \infty}{S_n} = S \\
        \Rightarrow \sum^\infty_{k=1} x_k \text{ сходится }
    \end{multline*}
    Мы так запаслись номерами, чтобы выражение было меньше $\varepsilon$
    

    Теперь ($\Leftarrow$). У нас кроме определения ничего нет. Возьмём какую-то фундаментальную последовательность. Откуда взять предел? Есть соотношения между элементами последовательности. Возьмём подпоследовательность, ведь у нас есть следствие \ref{cor:subseq-metric-sum-converges}! 
    Из свойств фундаментальных последовательностей, мы знаем, что существует подпоследовательность 
    \begin{gather*}
        \{x_{n_k} \}_{k=1}^\infty:||x_{n_1}|| + \sum_{k=1}^\infty ||x_{n_{k+1} - x_{n_k}}|| < +\infty \\
         \Rightarrow x_{n_1} + \sum^n_{k=1} (x_{n_{k+1}} - x_{n_k}) \text{~--- сходится, но:} \\
        S_m = x_{n_1} + \sum^{m-1}_{k=1}(x_{n_{k+1}} - x_{n_k}) = x_{n_m} \Rightarrow \exists S \in X \liml_{n \to \infty} x_{n_m} = S
    \end{gather*}
\end{proof}

\section{Пространства ограниченных функций}
\begin{definition}
    Пусть $A$~--- произвольное множество. Стандартное обозначение $m(A)$~--- множество всех ограниченных функций.
    \[ m(A) = \{ f: A \rightarrow \bR \text{ или } \bC, \sup_{x \in A} |f(x)| < \infty \} \]
\end{definition}

$f \in m(A), ||f||_{\infty} = \sup_{x \in A} |f(x)|$.

\begin{theorem}
    $(m(A), || \cdot ||_{\infty})$~--- банахово пространство
\end{theorem}
\begin{proof}
    Нужно проверить две вещи. Во-первых, что норма удовлетворяет аксиомам нормы. А во-вторых, что пространство с таким определением является полным.
     Просто по определению, никаких хитрых критериев. Возьмём
    фундаментальную подпоследовательность и покажем, что у нее есть предел.

    Проверяем, что $|| \cdot ||_\infty$ удовлетворяет аксиомам нормы.
    \[ ||f||_\infty = \sup_{x \in A} |f(x)| \geq 0, ||f||_\infty 0 \Rightarrow f(x) 0 \forall x \in A \text{ т.е. } f = \mathbb{0} \]

    $\lambda \in \bR$ (или $\bC$). $||\lambda f|| = \sup_{x \in A} |\lambda| \cdot ||f||_\infty$

    Нужно проверить неравенство треугольника.

    $f, g \in m(A)$. $x$~--- фиксированная точка в $A$

    $|f(x) + g(x)| \leq |f(x)| + |g(x)| \leq ||f||_\infty + ||g||_\infty \forall x \in A$
    \[\Rightarrow ||f+g||_\infty = \sup_{x \in A} |f(x) + g(x)| \leq ||f||_\infty + ||g||_\infty\]
    Теперь мы проверили аксиомы нормы. Доказываем полноту. 
    $\{f_n\}$~--- фундаментальная в $m(A)$.
    \[ \varepsilon > 0 \, \exists \, N \in \bN: (m > N \, \wedge \, n > N)  \Rightarrow ||f_n - f_m||_\infty < \varepsilon \text{ т.е. } \sup_{x \in A} |f_n(x) - f_m(x)| \]
    Первый вопрос: откуда взять претендента на роль предела? Еще желательно, чтобы он был единственный. Фиксируем $x$. Если для супремума есть неравенство, то и для
    $x$ тем более.
    $|f_n(x) - f_m(x)| < \varepsilon$ при $n,m > N$. 
     $\Rightarrow \{f_n(x)\}^\infty_{n=1}$~--- последовательность чисел в $\bC$ или $\bR$.
     
     $\Rightarrow \{f_n(x) \}^\infty_{n=1}$~--- фундаментальная $\Rightarrow \exists \liml_{n \to \infty} f_n(x)$

     \[ f(x) = \liml_{n \to \infty} f_n(x)  \forall x \in A \text {x фиксированный}\]
    \begin{gather*}
        |f_n(x) = f_m(x)| < \varepsilon \quad \text { пусть } m \to \infty \\
        \Rightarrow |f_n(x) - f(x)| \leq \varepsilon, x \in A \, \forall x \in A \\
        \Rightarrow ||f_n-f||_\infty = \sup_{x \in A} |f_n(x) - f(x)| \leq \varepsilon \text{ при } n > A
    \end{gather*}
     Последнее сображение, которое нужно добавить, это то, что $f$~--- элемент $A$. Мы можем записать $f$ как $f = (f - f_n) + f_n, f_n \in m(A), f-f_n \in m(A)$.

     \[\Rightarrow \liml_{n \to \infty} || f - f_n|| = 0 \Leftrightarrow \liml_{n \to \infty} f_n = f \in m(A) \]
 \end{proof}
 Давайте заметим, что у нас получилось определение равномерной непрерывности из математического анализа.

\[ \liml_{n \to \infty} f_n = f \in m(A) \Leftrightarrow \liml_{n \to \infty} \sup_{x \in A} |f_n(x) - f(x)| = 0 \Leftrightarrow f_n \darrow{A, n \to \infty}  f \] 

\begin{definition}[Топологический компакт]
    Множество $K$~--- топологический компакт, если оно обладает следующими свойствами
    \begin{enumerate}
        \item $\forall\{G_\alpha\}_{\alpha \in A}, G_\alpha$~--- открытые множества $K \subset \bigcup_{\alpha \in A} G_\alpha \exists \{\alpha_j\}_{j=1}^n, K \subset \bigcup^n_{j=1} G_{\alpha_j} $
        \item Хаусдорфовость 
        $\forall x,y (x \ne y) \in K \,\exists U,V$~--- открытые множества, $x \in U, y \in V, U \cap V = \varnothing $
    \end{enumerate}
\end{definition}

\begin{definition}
    $C(K) = \{ f: K \rightarrow \bR, f \text { непрерывна} \}$
    \[ ||f||_{C(K)} = ||f||_\infty = \sup_{x \in K} |f(x)| = \max_{x \in K} |f(x)| \]

\end{definition}

\begin{corollary}
    $K$~--- топологический компакт $\Rightarrow C(K)$~--- банахово
\end{corollary}

\begin{proof}
    $C(K) \subset m(K)$. $C(K)$~--- подпространство в алгебраическом смысле. Проверим, что $C(K)$~--- замкнуто в $m(K)$

    \[ \{ f_n \}, f_n = C(K), \liml_{n \to \infty} |f - f_n|_\infty = 0 \stackrel{замечание}{\Leftrightarrow} f_n \darrow{K, n \to \infty} f \stackrel{из анализа}{\Rightarrow} f \in C(K) \Rightarrow C(K) замкнуто \]
    тогда $m(K)$~--- полное и $C(K)$~--- полное.
\end{proof}


\section{Пространство последовательностей с $\sup$ нормой}

\begin{definition}
    $\bC^n, n \in \bN, l_n = \{x^\infty = (x_1, \ldots, x_n), x_j \in \bC \} $
    \[ ||x||_\infty =  \max_{1 \leq j \leq n} |x_j| \]

\end{definition}
$A = \{ 1, 2, \ldots, n \}, l_n^\infty = m(A) \Rightarrow l^\infty_n$~--- полное
Удобно думать, что последовательность~--- это функция на множестве натуральных чисел.

\begin{definition}[$l^\infty$]
    \begin{gather*}
        l^\infty = \{ X  = \{x_j\}_{j=1}^\infty, \sup_{j \in \bN} |x_j| < + \infty \} \\
        ||x||_\infty \sup_{j \in \bN} |x_j| \quad A = \{1, 2,3, \ldots, n, \ldots \} \\
        X = \seq{x}{j}_{j=1}^\infty \in m(A), \underset{f: A \rightarrow \bC}{f(j) = x_j} \\
        l^\infty := m(\bN) \Rightarrow l^\infty \text {~--- полное}
    \end{gather*}    
\end{definition}

 \begin{definition}
    \begin{gather*}
        c = \{ X = \seq{x}{j}^\infty_{j=1}, x_j \in \bC \quad \exists \liml_{n \to \infty}{x_n = x_0} \} \\
        c \subset l^\infty, ||x|| = ||x||_\infty = \sup ||X|| \\
        c_0 = \{x=\{x\}_{j=1}^\infty, \liml_{n \to \infty} x_j = 0\}, c_0 \subset c \subset l^\infty 
    \end{gather*}
    $c, c_0$~--- замкнутые подпространства в $l^\infty \Rightarrow c, c_0$~--- банаховы. 
 \end{definition}

 \section{Пространства $n$ раз непрерывно дифференцируемых функций на отрезке}

 \begin{definition}(норма $n$ производной)
    \[n \in \bN \quad C^{(n)}[a,b] = \{ f: [a,b] \rightarrow \bR \} \, \exists \, f^{(n)} \in C[a,b] \]
    \[ ||\underset{C[a,b]}{|f||_{(n)}}  = \max_{0 \leq k \leq n}, f^{0} = f \]
 \end{definition}

 \begin{theorem}B
    $C^{(n)}[a,b]$~--- банахово.
 \end{theorem}

 \begin{proof}
    \begin{gather*}
        \{f_m\}^\infty_{m=1} \text{~--- фундаментальная последовательность в } C^{(n)}[a,b] \\
        \varepsilon > 0 \exists N : (m > n \, \wedge \, q > n) \Rightarrow \normunder{f_m-f_q}{C^{(n)}} 
        < \varepsilon \Rightarrow \normunder{f_m^{(k)} < f_q^{(k)}}{\infty} < \varepsilon\\ 
         k = 0,1,\ldots,n 
    \end{gather*}

    \begin{multline}
        \{f_m^{(k)} \} \text {~--- фундаментальная в полном пространстве } C[a,b]\\
         \Rightarrow \exists \varphi_k \in C[a,b], f_n^{(k)} \darrow{[a,b]} \varphi_k, k = 0,1, \ldots, n \\
        \stackrel{\text{Анализ}}{\Rightarrow} (f_k^{(n)} \darrow{[a,b]} \varphi_0 \, \wedge \, \varphi_k^{0} \darrow{[a,b]} \varphi_1) \Rightarrow \varphi_1 = \varphi_0^\prime, \varphi_2 = \varphi_0^{\prime\prime}, \ldots, \varphi_n = \varphi_0^{(n)}
    \end{multline}
 \end{proof}

\end{document}
