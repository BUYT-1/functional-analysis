% !TeX root = ./document.tex
\documentclass[document]{subfiles}
\begin{document}
\chapter{Введение}
\section{qwe}
День рождения функционального анализа -- 1932 год. В этом году вышла книжка "Теория линейных операторов", автор -- С. Банах. Главная цель функционального анализа --
 изучение линейных операторов (но не только их).
Главным объектом у нас будет $X$ -- линейное топологическое пространство. Оно же линейное пространство над $\bC$ (или $\bR$).
Есть непрерывные операции
\begin{enumerate}
    \item $(x,z) \rightarrow x + z \quad x,z \in X$
    \item $(\alpha,x) \rightarrow \alpha x \quad \alpha \in \bC$
\end{enumerate}
Если у нас есть топологическое пространство, то у нас есть все любимые объекты из математического анализа -- пределы, непрерывность, производные, интегралы.

Пусть есть $X,Y$ -- линейные топологические пространства. Также есть линейное отображение $A: X \rightarrow Y$
\begin{definition}[Линейное отображение]
    \[ A(\alpha x + \beta z) = \alpha A x + \beta A z \]
\end{definition}

Если $\mathbb{dim} X < + \infty, \dim Y < + \infty$, то это линейная алгебра.

\[A: X \rightarrow X, \dim X = n, A = A^* \implies \exists \text {ОНБ} \{u_j\}_{j=1}^n \]
$\lambda_j$ -- собственные 
\[ A u_j = \lambda_j u_j \]
\begin{theorem}[Гильберт]
    $X$ -- гильбертово (сепарабельное) пространство.
    $A = A^* \quad A \cdot X \rightarrow X, \implies \exists$ ОНБ из собственных векторов.
\end{theorem}

Если $\dim Y = 1$, т.е. $Y = \bC$ (или $\bR$), то $A: X \rightarrow \bC$, $A$ -- линейный функционал.

$X$ - пространство функций, $f \in X$.

В математическом анализе мы изучаем $f \stackrel{?}{\implies} f^\prime$.
В функциональном анализе же у нас $X$ -- пространство функций, $f \in X$
\begin{gather}
    D(f) =  f^\prime \quad D: X \rightarrow Y
\end{gather}
и здесь мы задаемся вопросами о следующих свойствах $D(f)$
\begin{itemize}
    \item компактность
    \item самоспрояженность
    \item непрерывность
\end{itemize}


Отцы основатели функционального анализа:
\begin{itemize}
    \item Ф. Гильберт (1862 - 1943) Гильбертовы пространства
    \item С. Банах (1892 -1945) Банаховы пространства
    \item Ф.Рисс (1880-1956) пространства Лебега $L^p$
\end{itemize}
Ну и хочется еще упомянуть для вас, компьютер саентистов, отцов основателей кибернетики, который оставили немалый след в функциональном анализае
\begin{itemize}
    \item Н. Винер (1894-1964)
    \item Д. фон Нейман (1903 - 1957). Про его архитектуру, наверное, что-то слышали?
\end{itemize}

\section{Зачем изучать функциональный анализ}
Во-первых, он позволяет посмотреть на задачу с высокого уровня абстракции.

Рассмотрим пространство непрерывных функций $C[a,b]$, там введем норму $|f| = \max_{x \in [a,b]} | f(x)|. $ Рассмотрим пространство многочленов $P_n = \{ \sum^n_{k=0} a_kx^k, a_k \in \bR \}$
Существует ли такой многочлен, на котором инфимум достигается? И если да, то единственный ли он? 

\[ E_n (f) = \inf_{p \in \P_n} || f - p||  = \min_{p \in \P_n} ||f - p|| \]


На первый вопрос ответ да, это следует из общей теоремы функционального анализа. 
\[ \dim P_n = n + 1 < + \infty \]

На второй же вопрос ответ тоже да, и тут функциональный анализ не при чем. Суть в том, что у многочлена степени $n$ не может быть больше $n$ корней.

Ну и еще немаловажные причины
\begin{enumerate}
    \item язык функционального анализа -- междисциплинарный язык математики.
    \item его результаты применяются в математической физике, которая у нас будет в следующем семестре.
    \item это интересно и важно.  $0,1,2 = o(3)$.
    \item у нас будет экзамен, на котором придется говорить уже нам.
\end{enumerate}

Дополнительная литература по курсу. Первая Рассчитана на студентов. В некоторых местах рассказывается не так, как обычно пишут в книжках, а именно как придумать доказательство.
 Как прийти к тому, что требуется, а не в другую сторону, как обычно. Остальные же книги поумнее.
\
\begin{enumerate}
    \item А.Н.Колмогоров, С.В. Фомин "Элементы теории функций и Ф.А."
    \item М.Рид, Б. Саймон. 1 том "методы современной физики". Тонкая (можно осилить), рассказывается также про применение ФА.
    \item А.В. Канторович, Г.Г Акилов "Функциональный анализ". Похожа на энциклопедию. Но там можно найти всё.
    \item К. Итосида "Функциональный анализ".
    \item У. Рудин
\end{enumerate}

\end{document}