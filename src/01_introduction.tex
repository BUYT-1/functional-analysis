% !TeX root = ./document.tex
\documentclass[document]{subfiles}
\begin{document}
\part{Метрические пространства}
\chapter{Введение}
День рождения функционального анализа~--- 1932 год. В этом году вышла книжка <<Теория линейных операторов>>, автор~--- С. Банах. Главная цель функционального анализа~--- изучение линейных операторов (но не только их). Главным объектом у нас будет $X$~--- линейное топологическое пространство. Оно же линейное пространство над $\bC$ (или $\bR$).
Есть непрерывные операции\footnote{напоминание: вообще говоря, пространство~--- это структура, у которой есть множество ($X$~--- это оно) и две операции, но для упрощения повествования пространством далее называются множества, а операции подразумеваются}
\begin{enumerate}
    \item $(x,z) \rightarrow x + z \quad x,z \in X$
    \item $(\alpha,x) \rightarrow \alpha x \quad \alpha \in \bC$
\end{enumerate}
Если у нас есть топологическое пространство, то у нас есть все любимые объекты из математического анализа~--- пределы, непрерывность, производные, интегралы.

Пусть есть $X,Y$~--- линейные топологические пространства. Также есть линейное отображение $A: X \rightarrow Y$
\begin{definition}[Линейное отображение]
    $A: X \rightarrow Y$~--- линейное отображение $\Leftrightarrow$
    \[ \forall x \in X \forall z \in X \forall \beta \in \bC \forall \alpha \in \bC \, A(\alpha x + \beta z) = \alpha A x + \beta A z \]
\end{definition}

Если $\dim X < + \infty, \dim Y < + \infty$, то это линейная алгебра.

\[A: X \rightarrow X \land \dim X = n \land A = A^* \implies \exists \text {ОНБ} \{u_j\}_{j=1}^n \]
$\lambda_j$~--- j-е собственное число $\Leftrightarrow A u_j = \lambda_j u_j$
\begin{theorem}[Гильберт]
    $X$~--- гильбертово (сепарабельное) пространство,
    $A = A^*, A: X \rightarrow X \Rightarrow \exists$ ОНБ из собственных векторов.
\end{theorem}

Если $\dim Y = 1$, т.е. $Y = \bC$ (или $\bR$), то $A: X \rightarrow \bC$, $A$~--- линейный функционал.

В математическом анализе мы изучаем $f: \bC \rightarrow \bC$.
В функциональном анализе же у нас $X$~--- пространство функций, $f \in X$
   \[ D(f) =  f^\prime \quad D: X \rightarrow Y \]
и здесь мы задаемся вопросами о следующих свойствах $D(f)$
\begin{itemize}
    \item компактность
    \item самосопряжённость
    \item непрерывность
\end{itemize}


Отцы-основатели функционального анализа:
\begin{itemize}
    \item Ф. Гильберт (1862--1943) Гильбертовы пространства;
    \item С. Банах (1892--1945) Банаховы пространства;
    \item Ф.Рисс (1880--1956) пространства Лебега $L^p$.
\end{itemize}
Ну и хочется ещё упомянуть для вас, компьютер саентистов, отцов основателей кибернетики, которые оставили немалый след в функциональном анализе
\begin{itemize}
    \item Н. Винер (1894--1964);
    \item Д. фон Нейман (1903--1957). Про его архитектуру, наверное, что-то слышали?
\end{itemize}

\section{Зачем изучать функциональный анализ}
Во-первых, он позволяет посмотреть на задачу с высокого уровня абстракции.

Рассмотрим пространство непрерывных функций $C[a,b]$, там введём норму $|f| = \max_{x \in [a,b]} | f(x)|.$ Рассмотрим пространство многочленов $P_n = \{ \sum^n_{k=0} a_kx^k, a_k \in \bR \}$

\[ E_n (f) = \inf_{p \in P_n} || f - p||  = \min_{p \in P_n} ||f - p|| \]

Существует ли такой многочлен, на котором инфимум достигается? И если да, то единственный ли он? 

На первый вопрос ответ да, это следует из общей теоремы функционального анализа. 
\[ \dim P_n = n + 1 < + \infty \]

На второй же вопрос ответ тоже да, и тут функциональный анализ ни при чём. Суть в том, что у многочлена степени $n$ не может быть больше $n$ корней.

Ну и ещё немаловажные причины
\begin{enumerate}
    \item язык функционального анализа~--- междисциплинарный язык математики;
    \item его результаты применяются в математической физике, которая у нас будет в следующем семестре;
    \item это интересно и важно.  $0,1,2 = o(3)$;
    \item у нас будет экзамен, на котором придется говорить уже нам.
\end{enumerate}

Дополнительная литература по курсу. Первая рассчитана на студентов: в некоторых местах рассказывается, как придумать доказательство, как прийти к тому, что требуется,
а не в обратную сторону, как обычно. Остальные же книги поумнее.
\
\begin{enumerate}
    \item А.Н.Колмогоров, С.В. Фомин <<Элементы теории функций и Ф.А.>>;
    \item М.Рид, Б. Саймон. 1 том <<методы современной физики>>. Тонкая (можно осилить), рассказывается также про применение ФА;
    \item А.В. Канторович, Г.Г Акилов <<Функциональный анализ>>. Похожа на энциклопедию. Но там можно найти всё;
    \item К. Итосида <<Функциональный анализ>>;
    \item У. Рудин <<Функциональный анализ>>.
\end{enumerate}

\end{document}
