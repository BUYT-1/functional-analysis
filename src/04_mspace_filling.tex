% !TeX root = ./document.tex
\documentclass[document]{subfiles}
\begin{document}
\chapter{Пополнение метрического пространства}
Мы привели несколько примеров нормированных пространств, не являющихся полными. Приведём еще один пример.
\begin{definition}
    \[ \Rho = \left\{ p(x) = \sum^n_{k=0}a_k x^k, a_k \in \bR, n \geq 0 \right\} \]
\end{definition}
$\Rho$ (подпространство в алгебраическом смысле) $ \subset C[a,b], \, ||p||_\infty = \max_{x \in [a,b]} |p(x)|$
$e^x \notin \Rho$, $p_n(x) = \sum^n_{k=0} \frac{x^k}{k!}, \Rightarrow p_n \darrow{[a,b], n \to \infty} e^x$
это не многочлен, потому что если сколько-то раз продифференцировать многочлен, он станте тождественным 0
$\Rightarrow \overline{\Rho} \setminus \Rho \ni e^x \Rightarrow \Rho$~--- не замкнуто $\Rightarrow \Rho$~--- не полное.

\[ \overline{\Rho} = C[a,b] \]
\begin{theorem}[Вейерштрасса, 1885]
    $f \in C[a,b], \forall \varepsilon > 0 \, \exists p \in \Rho$ т.ч.  $||f - p|| < \varepsilon$
    (любую функцию на отрезке можно приблизить многочленами)
\end{theorem}

% тту рисунок отрезок в области G открытой, какую бы пилу не нарисовали, ее можно приблизить многочленами$
\[ p_n \darrow{G} f \Rightarrow f \text{ аналитическая в } G \]

\section{Пополнение метрического пространства}
Несколько простых свойств метрики, и все следуют из неравенства треугольника
\begin{theorem}[Свойства метрики]
    $(X, \rho)$~--- метрическое
    \begin{enumerate}
        \item $x,y,z,u \in X \quad \Rightarrow |\rho(x, u) - \rho(y, z)| \leq \rho(x,y) + \rho(u,z)$
        \item $\rho: X \times X \rightarrow \bR \Rightarrow \rho(x,y)$~--- непрерывная функция
        \item $A \subset X, A $~--- подмножество, $\rho(x,A) = \inf_{y \in A} \rho(x,y) \Rightarrow \rho(x,A)$~--- непрерывная функция от $x$
        \item $A \subset X, A = \overline{A}, x_0 \notin A \Rightarrow \rho(x_0, A) > 0 $
    \end{enumerate}
    
\end{theorem}
\begin{proof}
    \begin{enumerate}
        \item $\rho(x,u) \leq \rho(x,y) + \rho(y,u) \leq \rho(x,y) + \rho(y,z) + \rho(z,u) \Rightarrow \rho(x,u) - \rho(y,z) \leq \rho(x,y) + \rho(z,u)$
        Аналогично $\rho(y,z) - \rho(x,u) \leq \ldots$
        из всего $\Rightarrow 1)$
        \item Докажем непрерывность с помощью последовательности. \\
         $\rho(x,y)$~--- непрерывная?
         \[ \liml_{n \to \infty} x_n = x, \liml_{n \to \infty} y_n = y \Leftrightarrow \liml_{n \to \infty} \rho(x_n, x) = 0 = \liml_{n \to \infty} \rho(y_n, y) \]
         $\rho(x,y) - \rho(x_n, y_n)| \stackrel{(1)}{\leq} \rho(x, x_n) + \rho(y, y_n) \underset{n \to \infty}{\longrightarrow} 0 \Rightarrow \liml_{n \to \infty} \rho(x_n, y_n) = \rho(x,y)$
         \item $A \subset X, \, x,z \in X, \, |\rho(x,A) - \rho(z,A)| \leq ? $ \\
         Пусть $y \in A$
         \begin{multline*}
            \rho(x,y) \leq \rho(x,z) + \rho(z,y) \Rightarrow \rho(x,A) \leq \rho(x,z) + \rho(z,y) \, \forall y \in A \\
            \Rightarrow \rho(x,A) \leq \rho(x,z) + \inf_{y \in A} \rho(z,y) = \rho(x,z) + \rho(z,A) \Rightarrow \\
            \rho(x,A) - \rho(z,A) \leq \rho(x,z)
         \end{multline*}
         Но нам нужен модуль. Можем сказать, что $x$ и $z$ ничем не отличаются, аналогично $\rho(z,A) - \rho(x,A) \leq \rho(x,z) \Rightarrow 3 $
         \item  \[ x_0 \notin A \Rightarrow x_0 \in X \setminus A \text{ открытое} \]
         $\Rightarrow \exists \delta > 0 \quad B_\delta(x_0) \subset X \setminus A \Rightarrow \rho(x_0, A) \geq \delta$
    \end{enumerate}

\end{proof}
Перед определением пополнения нам потребуется несколько определений, связанных с отображениями в метрических пространствах.

$(X, \rho), (Y, d)$~--- метрические простарнства. $T: X \rightarrow Y$.
\begin{definition}[Изометрическое вложение]
    \[ d(T_x, T_z) = \rho(x,z) \quad \forall x,z \in X \]
Обозначение: $ X \hookrightarrow Y$
\end{definition}

\begin{definition}[Изометрия]
    $T$~--- изометрическое вложение, $T(X) = Y$
    
\end{definition}
\begin{definition}[Изометричность пространств]
    $(X, \rho), (Y,d)$ изометричны, если $\exists T: X \rightarrow Y, T$~--- изометрия
\end{definition}

\begin{property}
    $T$~--- изометрическое вложение $\Rightarrow$ $T$~--- инъективное, непрерывное
\end{property}
\begin{proof} 
    $x, z \in X, T: X \rightarrow Y,$ пусть $T_x = T_z \Rightarrow d(T_x, T_z) = 0$ 
    Значит, исходное расстояние тоже 0 по свойству метрики. $d(x,z) = 0 \Rightarrow x = z$

    Инъективность проверили, теперь непрерывность, это еще проще.
    \[ \liml_{n \to \infty} = x \Leftrightarrow \liml_{n \to \infty} \rho(x, x_n) = 0 \Rightarrow \liml_{n \to \infty} d(T_{X_n}, T_x) = 0 \Rightarrow \liml_{n \to \infty} T_{X_n} = T_x \]

\end{proof}

\begin{property}
    Если $T$~--- изометрия, то $\exists T^{-1}$~--- изометрия.
\end{property}

\begin{property}
    <<Изометричность>>~--- отношение эквивалентности на множестве метрических пространств
\end{property}

И наконец

\begin{definition}[Пополнение м. пространства]
    $(X, \rho)$~--- метрическое пространство. $(Z, d)$~--- полное метрическое пространство.
    $(Z,d)$~--- пополнение $(X, \rho)$, если существует $T: X \rightarrow Z$
    \begin{enumerate}
        \item $T$~--- изометрическое вложение
        \item $\overline{T(X)} = Z$
    \end{enumerate}
\end{definition}

\begin{remark}
    Не обязательно искать пространство, удовлетворяющее и второму свойству. Достаточно найти такое, которое удовлетворяет первому.
    $(X,\rho)$~--- метрическое пространство, $(U,d)$~--- полное метрическое пространство. Пусть $\exists T: X \rightarrow U$~--- изометрическое вложение. Если 2 свойство не выполняется, то легко такое $Z$ построить.
     Возьмём замыкание образа.
     $ Z = \overline{T(X)} \Rightarrow (Z,d)$~--- пополнение X.
\end{remark}
Теперь обещанная теорема. Возьмём любое метрическое пространство и покажем, что у него есть пополнение.
\begin{theorem}[О пополнении метрического пространства]
    $(X, \rho)$~--- метрическое $\Rightarrow \, \exists$ пополнение $(Z,d)$
\end{theorem}

\begin{proof}
    Есть классическое доказательство с рассмотрением всех фундаметнальных последовательностей, рассмотрением фактор-пространства, 
    муторным разбором случаев. Мы пойдем другим путём. 
    Будет короткое, но \textbf{фантастически}
    непонятное доказательство в том смысле, что непонятно, как его придумать. \\
    Мы собираемся использовать $m(X) = \{ f: X \rightarrow \bR, \sup_{x \in X} |f(x)| < + \infty \} $
    \[ ||f||_{m(X)} = ||f||_\infty = \sup_{x \in X} |f(x)| \]
    $m(X)$~--- полное пространство. \\
    Каждой точке мы сопоставим функцию. Вот такая идея!
    $\varphi: X \rightarrow m(X)$.
    Оно же будет изометрическим вложением, то есть будет сохранять расстояния.  \\
    Сначала будет маленькое облегчающее предположение про $X$, от которого мы потом откажемся.
    Пусть $X$~--- ограниченное, то есть $\exists \, M > 0$ т.ч. $\forall x,y \in X \, \rho(x,y) \leq M$.
    Единственная цель предположения~--- формула для $\varphi$ будет чуть проще. Вообще, можно было бы обойтись и без него. \\
    $t \in X, t$~--- фиксирован, $f_t(x) = \rho(x,t)$.
    При фиксированном $t$~--- это функция на $X$. Именно сюда наше отображение будет отображать $t$. Одной точке~--- целая функция, понятно?

    \[ \varphi(t) := f_t(x) \text{ т.е. } \varphi: t \rightarrow f_t(x) \]
    \[ |f_t(x)| \leq M \Rightarrow f_t \in m(X) \]
    Самое главное. Проверим, что отображение сохраняет расстояния. Это очень легко. Возьмём 2 точки.
    \begin{gather*}
        \text{ Пусть } t, s \in X, \quad ||f_t - f_s||_\infty = \sup_{x \in X} |\rho(x,y) - \rho(x,s)|\\
        |\rho(x,t) - \rho(x,s) | \leq \rho(t,s), \, \text { Пусть } x = t \Rightarrow |\rho(t,t) - \rho(t,s) | = \rho(t,s)
        %Естесвенно, с таким же успехом можно было взять $x = s$
    \end{gather*}
    \[ \Rightarrow  || \varphi(t) - \varphi(s)||_\infty = \rho(t,s) \Rightarrow \varphi \text{~--- изометрическое вложение} \]

    Посмотрим, что будет, если откажемся от этого облегчающего предположения. Надо будет чуть исправить отображение $\varphi$.
    $X$~--- любое метрическое пространство. $a \in X$~--- фиксированная точка.\\
    $t \in X, f_t(x) = \rho(x,t) - \rho(x,a) \Rightarrow |f_t(x)| \leq \rho(a,t) \Rightarrow f_t \in m(X) $

    Раньше мы могли так брать и не вылетать из пространства из-за ограниченности. Вычтем эту штуку, чтоыб попасть, куда надо. \\
    \[ t,s \in X \Rightarrow f_t(x) - f_s(x) = \rho(x,t) - \rho(x,s) \stackrel{(1)}{\Rightarrow} 
        ||f_t - f_s||_\infty = \rho(s,t) 
        \]
    Пополнение $X$: $ \overline{\varphi(X)}^{|| \cdot ||_\infty} = Z, (Z, || \cdot ||_\infty) $
\end{proof}

Таким образом, изучение метрических пространств можно свести к изучению подмножества пространства непрерывных функций.
\begin{remark}
    Забегая далеко вперёд.
    $(X, || \cdot ||)$~--- нормированное, $X^*$~--- множество непрерывных линейных функционалов на $X$, $X^*$~--- полное (ВСЕГДА).

    Мы построим каноническое вложение $\pi: X \rightarrow \underbrace{(X^*)^*}_{\text{полное}}$,
    $\overline{\varphi(x)}^{X^{**}}$~--- пополнение X.
\end{remark}


\section{Теорема о вложенных шарах}

Когда-то в анализе была теорема Кантора о том, что если есть последовательность вложенных друг в друга отрезков, то их пересечение не пусто. Мы докажем похожее утверждение для метрических пространств. Оказывается, то утверждение
было связано с полнотой вещественной прямой $\bR$.
$(X, \rho)$~--- метрическое пространство, $r > 0, x \in X$ \\
Введём стандартное обозначение замкнутого шара. 

\[ D_r(x) = \{ y \in X: \rho(x,y) \leq r\}\]

\begin{theorem}[О вложенных шарах]
    $(X, \rho)$~--- метрическое пространство. $X$~--- полное $(|\Leftrightarrow (\forall \{D_n \}^\infty_{n=1}, D_n = D_{r_n}(x_n))$, $D_{n+1} \subset D_n, 
    \liml_{n \to \infty} r_n = 0 \Rightarrow \bigcap^{+\infty}_{n=1} D_n \neq \varnothing$).
    По сранению с теоремой Кантора у нас есть дополнительное предположение о стремлении к нулю, которое здесь важно, а на прямой было как данность.
\end{theorem}

\begin{proof}
    $\Rightarrow \quad X$~--- полное  \\
    \[ \{D_n\}^\infty_{n=1}, D_n = D_{r_n}(x_n), D_{n+1} \subset D_n, \liml_{n \to \infty} r_n = 0  \]
    Надо проверить, что центры шаров образуют фундаментальную последовательность, то есть что $\seq{x_n}^\infty_{n=1}$~--- фундаментальная. \\
    Пусть $\varepsilon > 0 \quad \exists \, N \in \bN \quad r_n < \varepsilon$ при $n \geq N$.
    \begin{multline*}
        (n > N \, \wedge \, m > N) \Rightarrow (x_n \in D_n \, \wedge \, x_m \in D_n) \Rightarrow \rho(x_n, x_m) \leq  \\ 
        \leq \rho(x_n, x_N) + \rho(x_m, x_N) \leq 2 \varepsilon
    \end{multline*}
    \begin{gather*}
        X \text{~--- полное } \Rightarrow \exists \liml_{n \to \infty} x_n = x \\
        \text{ любое фиксированное } m \in \bN \quad x_n \in D_m \, \forall n \geq m, D_m \text{~--- замкнутое } \\
        \Rightarrow \liml_{n \to \infty, n \geq m} x_n = x \in D_m \\
        \Rightarrow x \in \bigcap^\infty_{m=1} D_m
    \end{gather*}
    $\Leftarrow$ \\
    Ничего кроме определения для доказательства полноты у нас нет. Пусть $\seq{x_n}^\infty_{n=1}$~--- фундаментальная.
    Возьмём достаточно быстро убывающую последовательность $\varepsilon_k = \frac{1}{2^k}$. Существует $\seq{x_{n_k}}^\infty_{k=1}, \rho(x_{n_k}, x_{n_{k+1}}) < \frac{1}{2^{k+1}}$. \\
    $D_k = D_{\varepsilon_k} (x_{n_k})$
    %тут рисуночек что происходит с шаром%

    \begin{gather*}
        \begin{cases}
            y \in D_{k+1} \Rightarrow \rho(y, x_{n_{k+1}}) \leq \frac{1}{2^{k+1}} \\
            \rho(x_{n_k}, x_{n_{k+1}}) < \frac{1}{2^{k+1}}
        \end{cases} \Rightarrow 
    \end{gather*}
    \begin{multline*}
        \rho(y, x_{n_k}) \leq \rho(y, x_{n_{k+1}}) + \rho(x_{n_{k+1}}, x_{n_k}) < \frac{1}{2^{k+1}} + \frac{1}{2^{k+1}} = \frac{1}{2^k} \\
        \Rightarrow y \in D_k \Rightarrow D_{k+1} \subset D_k
    \end{multline*}
    Мы взяли произвольный элемент из $D_{k+1}$ и показали, что он принадлежит $D_k$, то есть показали вложенность элементов последовательности.

    \[ \Rightarrow \exists x \in \bigcap^\infty_{k=1} D_k \quad \rho(x, x_{n_k}) \leq \frac{1}{2^k} \Rightarrow \liml_{k \to \infty} x_{n_k} = x \]

    По свойству фундаметнальных последовательностей из первой лекции $\liml_{n \to \infty} x_n = x$
\end{proof}

\begin{remark}
    В условиях теоремы пересечение вложенных шаров $\bigcap^\infty_{n=1} D_n$ состоит из одной точки.
\end{remark}

\begin{proof}
    Пусть $x \in \bigcap^\infty_{n=1} D_n, \Rightarrow \rho(x, x_n) \in r_n, \liml_{n \to \infty} r_n = 0 \Rightarrow \liml_{n \to \infty} x_n = x$.
    А мы знаем, что предел в метрическом пространстве единственный.
\end{proof}

\begin{remark}
    Условие, что $\liml_{n \to \infty} r_n = 0$ в теореме существенно.
\end{remark}

\begin{example}[Замкнутые множества]
    $\{ F_n \}^\infty_{n=1}, F_n$~--- замкнутое, $F_{n+1} \subset F_n, F_n \subset \bR, \bigcap^\infty_{n=1} F_n = \varnothing, F_n = [n, + \infty)$
\end{example}


\begin{example}[По теореме]
    \[ X [1. +\infty) \quad \rho(x,y) = \begin{cases}
        1 + \frac{1}{x+y}, \quad x \ne y \\
        0, \quad x = y
    \end{cases} \]
    Проверим, что $\rho$~--- метрика. $x,y,z$
    \[ \rho(x,y) + \rho(y,z) = 1 + \frac{1}{x+y} + 1 + \frac{1}{y+z} > 1 + 1 > 1 + \frac{1}{x+z} = \rho(x,z) \]
    Проверяем полноту. Пусть $\seq{x_n}^\infty_{n=1}$ фундаментальная, $\varepsilon = \frac{1}{2} \Rightarrow$ 
    \begin{gather*}
        \exists \, N \in \bN : (n \geq N \, \wedge \, m \geq N) \rho(x_n, x_m) < \frac{1}{2} \Rightarrow \left(\rho(x_n, x_N) < \frac{1}{2} \, \wedge \, \rho(x_m, x_N) < \frac{1}{2}\right) \Rightarrow \\
        x_N = x_{N+1} = x_{N+2} = \ldots \\
        \Rightarrow  \, \exists \liml_{n \to \infty} x_n = X_N \Rightarrow (X, \rho) \text{~--- полное}
    \end{gather*}
    Полноту проверили. \\
    $r_n = 1 + \frac{1}{2n}, x_n = n; D_n = D_{r_n}(n), h \in D_n$. Пусть $x \ne n, x \in D_n \Rightarrow \rho(x, x_n) = 1 + \frac{1}{x+n} \leq 1 + \frac{1}{2n}$
\end{example}

\begin{remark}[Домашнее задание]
    Если $(X, || \cdot ||)$~--- банахово, то $D_{n+1} \subset D_n \{ D_n \}^\infty_{n=1} \Rightarrow \bigcap^\infty_{n=1} D_n \ne \varnothing$  (требование $\liml_{n \to \infty} r_n = 0$ лишнее)
\end{remark}

\section{Сепарабельные пространства}

$(X, \rho)$~--- метрическое пространство,
\begin{definition}[$A$ плотно в $C$]
     $A \subset X, C \subset X$. $A$ плотно в $C$, если $C \subset \overline{A} \Leftrightarrow$
    \[ \forall x \in C \, \forall \varepsilon > 0 \, \exists a \in A \, \rho(x,A) < \varepsilon \Leftrightarrow \forall \varepsilon > 0 \, C \subset \bigcup_{a \in A} B_{\varepsilon}(a) \]
    Любой элемент $C$ можно сколь угодно хорошо приблизить элементами из $A$.
\end{definition}

\begin{definition}[$A$ всюду плотно в $C$]
    $A$~--- всюду плотно в $X$, если $\overline{A} = X$
\end{definition}

Чем же полезно это свойство? Если хотят доказать свойство для $X$, то часто доказывают сначала для всюду плотного 
подмножества.

\begin{definition}[Сепарабельное пространство]
    $(X, \rho)$~--- \textbf{сепарабельное}, если $\, \exists E \subset X, E = \seq{x_n}_{n=1}^\infty, \overline{E} = X$
\end{definition}

\begin{theorem}
    $n \in \bN, q \leq p \leq +\infty$, 
    \[l^p_n \text{~--- сепарабельное} \]
\end{theorem}

\begin{proof}
    \begin{gather*}
        l^p_n = ( \bR^n, || \cdot ||_p) = \{ x = (x_1, \ldots, x_n), x_j \in \bR, ||x||_p \} \\
        E = \bQ^n = \{ x = (x_1, \ldots, x_n), x_j \in \bQ \}
    \end{gather*}
    Если $ (\bC^n, || \cdot ||_p), \tilde{\bQ} = \{ x + iy, \, x, y \in \bQ \}, E = \tilde{\bQ}^n$
\end{proof}

\begin{theorem}
    $F$~--- финитные последовательности, $1 \leq p \leq +\infty$
    \[ (F, || \cdot ||_p) \text{~--- сепарабельно} \]
\end{theorem}
\begin{proof}
    $E = \bigcup^\infty_{n=1}, \, \bQ^n = \{ x = (x_1, x_2, \ldots, x_{N(x)}, 0, 0, \ldots, ), x_j \in \bQ \}$.
    Попросту говоря, все финитные последовательности, координаты которых рациональны.
\end{proof}

\begin{theorem}
    $l^p, 1 \leq p < + \infty, C_0$~--- сепарабельные
\end{theorem}
\begin{proof}
    На прошлой лекции мы доказали, что 
    \begin{gather*}
        (F, ||\cdot||_p), \overline{F}^{||\cdot||_p} = l^p \text{ при } 1 \leq p < + \infty \\
        \begin{cases}
             E = \bigcup^\infty_{n=1} \bQ^n \text{~--- всюду плотно в } F \\
             F \text{~--- всюду плотное в } l^p \end{cases} \Rightarrow  \\
             E \text{ всюду плотно в } l^p, 1 \leq p < + \infty 
    \end{gather*}
    Почему любой элемент из $l^p$ может быть приближен финитной последоватностью? Мы ее просто отрезаем.
\end{proof}

Ответ на упражнение для читателя, которое было на прошлой лекции: 
$F$~--- подпространство в алгебраическом смысле, $F \subset l^\infty$, $\overline{F}^{|| \cdot ||_\infty} = C_0$
\begin{gather*}
    x_0 \in C_0 \Leftrightarrow x = \seq{x_n}^\infty_{n=1}, \liml_{n \to \infty} x_n = 0 \\
    \text{ берем первые } m \text { координат и дополняем их нулями } \\
    x^{(m)} = (x_1, \ldots, x_m, 0, 0, \ldots, 0, \ldots) \Rightarrow x^{(m)} \in F \\
    ||x - x^{(m)}||_\infty = \sup_{k > m} |x_k| \underset{m \to \infty}{\longrightarrow} 0
\end{gather*}

Остаётся вопрос, почему $C_0$~--- замкнутое множество. Можно в лоб, а можно по-учёному рассудить.
\begin{gather*}
    \text{ пусть } \seq{y^{(m)}}^\infty_{m=1}, y^{(m)} \in C_0, \, y^{(m)} \underset{m \to \infty}{\longrightarrow} y \text{ в } C_0 \\
    \Rightarrow \liml_{m \to \infty} ||y - y^{(m)}||_\infty = 0 \qquad y = \{ y_n \}^\infty_{n=1}, \liml_{n \to \infty} y_n = 0 \text{ ???}
\end{gather*}
А это равномерная сходимость на множестве натуральных чисел, то есть это тот случай, когда можно менять местами пределы.
\[ \liml_{n \to \infty} y_n = \liml_{m \to \infty} \underbrace{\liml_{n \to \infty} y_n^{(m)}}_{=0} = 0 \]

Упражнение: $C$~--- сепарабельное, $C \subset l^\infty$ 

\begin{theorem}
    $l^\infty$~--- не сепарабельное
\end{theorem}

Какой бы шарик из $X$ мы бы не предъявили, там всегда будет элемент всюду плотного множества.

\begin{proof}
    \[
        A \subset \bN \quad X^A_n = \begin{cases}
            1, n \in A \\
            0, n \notin A
        \end{cases}\]

    Мощность $\{A, A \subset \bN \}$~--- континуум (> счётное). Это и будет центр пересекающихся шариков. Посмотрим, каким будет расстояние между двумя разными точками.
    \begin{gather*}
        A \subset \bN, C \subset \bN, A \notin \bC \\
        X^A_n - X^c_n = \begin{cases}
            1 \\
            0 \\
            -1
        \end{cases} \Rightarrow ||x^A - x^C||_\infty = \sup_{n \in \bN} |X^A_n - ^C_n| = 1 
    \end{gather*}
    То есть если 2 множества не равны, то расстояние между ними~--- единица.

    \[B_{\frac{1}{2}} (x^A) \cap B_{\frac{1}{2}}(x^C) = \varnothing \] 
    Мы предъявили несчётный набор дизъюнктных шариков. $E$~--- всюду плотно в $l^\infty \Rightarrow \, \forall A \subset \bN \, \exists e_A \in B_{\frac{1}{2}}(x^A)$

    \[ A \ne C \Rightarrow e_A \ne e_C, \qquad \underbrace{\seq{e_A}_{a \subset \bN}}_{\text{ несчётно }} \subset E \Rightarrow E \text{ несчётно } \]
    То, что у всех шариков одинаковый радиус~--- это просто приятный бонус.
\end{proof}

\begin{theorem}
    $(X, \rho)$~--- сепарабельное, $ Y \subset X \Rightarrow (Y, \rho)$~--- сепарабельное. 
\end{theorem}
\begin{proof}
    $\exists \, E = \seq{x_n}^\infty_{n=1}$~--- всюду плотно в $X$, $x_0 \in X$
    \begin{gather*}
    \rho(x_n, Y) = \inf_{y \in Y} \rho(x_n, y) \Rightarrow \\
    \exists \, \seq{y_{n,k}}^\infty_{k=1} \quad \liml_{k \to \infty} \rho(x_n, y_{n,k}) = \rho(x_n, Y) \\ 
    y_{n,k} \in Y, \, F = \seq{y_{n_k}}_{n,k} \text{~--- счётное }, F \subset Y    
    \end{gather*}
    Проверим, что $F$~--- всюду плотно в $Y$. Пусть $y \in Y, \varepsilon > 0 \Rightarrow \, \exists x_n : \rho(y, x_n) < \varepsilon$.
    Из этого неравенства мы делаем вывод, что $\rho(x_n, Y) < \varepsilon$. Значит, $\exists k: \rho(x_n, y_{n,k}) < \varepsilon \Rightarrow$
    \[ \rho(y, y_{n,k}) \leq \rho(y, x_n) + \rho(x_n, y_{n,k}) < \varepsilon + \varepsilon = 2 \varepsilon \]
\end{proof}

\begin{corollary}
    $X$~--- бесконечное множество $\Rightarrow m(X) $~--- не сепарабельное.  
\end{corollary}
\begin{proof}
    Можно слово в слово повторить доказательство для $l^\infty$, но мы воспользуемся последними доказанными теоремами.
    \begin{gather*}
        \exists \seq{a_j}^\infty_{j=1}, a_j \in X, a_j \ne a_i \text{ при } i \ne j \\
        Y = \{ f \in m(X), f(x) = 0 \text{ если } x \ne a_j \} \sup_{j \in \bN} |f(a_j)| < + \infty \\
        Y \text{ изометрично } l^\infty, f \in Y, T(f) = \seq{f(a_j)}^\infty_{j=1} \in l^\infty \\
        Y \text{~--- не сепарабельно } \Rightarrow \text{ и по последней теореме } \\
        m(X) \text{~--- не сепарабельно}
    \end{gather*}
\end{proof}

\begin{theorem}
    \[ C[a,b] \text {~--- сепарабельно} \]
\end{theorem}
\begin{proof}[1 часть]
    \begin{gather*}
        L = \{ \text{ ломаные } \} \quad a = x_0 < x_1 < \ldots < x_n = b \quad \seq{y_k}^n_{k=0}, y_k \in \bR \\
        L(x) \text{~--- ломаные} \\
        L(x_k) = y_k, \, k = 0,1, \ldots, n \quad l(x) \text{ линейная на } [x_k, x_{k+1}]
    \end{gather*}
    Отметим, что $L$~--- всюду плотное множество в пространстве непрерывных функций. Это связано с равномерной непрерывностью. 
    Никаких надежд на то, что оно будёт счётным нет.
    \begin{gather*}
        \text{ пусть } f \in C[a,b], \, \varepsilon > 0 \Rightarrow \exists \delta > 0 : |x-y| < \delta \\
        \Rightarrow |f(x) - f(y)| < \varepsilon \\
        \exists \seq{x_k}^n_{k=0} \text{~--- разбиение } \quad x_{k+1} - x_k < \delta \\
        y_k := f(x_k) \quad L(x) \text{~--- ломаная } \\
        \Rightarrow |f(x) - L(x)| < \varepsilon \Rightarrow ||f-L||_\infty \leq \varepsilon \Rightarrow \overline{\calL} = C[a,b] \\
    \end{gather*}
    как сделать так, чтобы множество ломаных было счётным? возьмём в качестве вершин элементы $\bQ$
    \begin{gather*}
        E = \{ L \in \calL, \, x_k, y_k \in \bQ \} \text{~--- счетное множество } \\
        \begin{cases}
            \calL \subset \overline{E} \\
            \overline{\calL} = C[a,b]
        \end{cases} \Rightarrow E \text{~--- всюду плотно, т.е. } \overline{E} = C[a,b]
    \end{gather*}
    
\end{proof}

\begin{proof}[2 часть]
    по т. Вейерштрасса замыкание многочленов~--- тоже пространство непрерывных функций.
    \begin{gather*}
        \Rho = \{ p(x) = \sum^n_{k=0} a_k x^k \} \quad \overline{\Rho} - C[a,b] \\
        E = \{ p(x) = \sum^n_{k=0} a_k x^k, \, a_k \in \bQ \} \\
        \begin{cases}
            \Rho \subset \overline{E} \\
            \overline{\Rho} = C[a,b]
        \end{cases} \Rightarrow \overline{E} = C[a,b]
    \end{gather*}
\end{proof}


\section{Нигде не плотные множества}

\begin{definition}
    $(X, \rho)$~--- метрическое пространство. $A \subset X, A$~--- \textbf{ нигде не плотно} в $X$, если 
    \[ \forall B_r(x) \text { при } r > 0, x \in X \quad B_r(x) \not \subset \overline{A} \Leftrightarrow \Int(\overline{A}) = \varnothing \Leftrightarrow \]
\end{definition}
Если мы рассмотрим замыкание, никакого шарика там не будет. Иначе: если мы рассмотрим внутренность замыкания, она будет пустой.

\begin{multline*}
    \forall r > 0, x \in X \quad B_r(x) \, \exists B_{r_1}(x_1) \subset B_r(x), B_{r_1}(x_1) \cap A = \varnothing \\
    \Leftrightarrow \forall r > 0, x \in X D_r(x) \, \exists D_{r_1}(x_1) \subset D_r(X), D_{r_1}(x_1) \cap A = \varnothing
\end{multline*}

Скоро докажем связь между нигде не плотными множествами и полными пространствами.
 Но сперва определение, которое не будет часто встречаться, но сам факт~--- полезный.
\begin{definition}[множество первой категории]
    $M \subset X, (X, \rho)$. $M$~--- \textbf{ множество первой категории}, если 
    \[ M = \bigcup^\infty_{j=1} E_j, E_j \text{ нигде не плотно в } X \]
\end{definition}

$M$~--- \textbf{ множество второй категории}, если $M$ нельзя представить в виде объединения счетного числа нигде не плотных множеств.

\begin{theorem}[Бэр, о категориях]
    $(X, \rho)$~--- полное $\Rightarrow X$~--- множество второй категории.
\end{theorem}

\begin{proof}
    Можно было бы даже от противного. Но мы возьмём семейство $\seq{M_j}^\infty_{j=1}$, $M_j$~--- нигде не плотно в $X$, $\, E - \bigcup^\infty_{j=1} M_j$.
      Мы докажем, 
    что найдётся хоть одна точка, которая принадлежит $X$ и не принадлежит $E$. Это и будет обозначать, что $X$ невозможно представить
    в виде такого объединения.
    \begin{gather*}
        x_0 \in X \quad D_0 = \{ y: \rho(x_0, y) \leq 1\} \\
        M_1 \text{~--- нигде не плотно } \Rightarrow \, \exists D_1 = D_{r_1}(x_1) \subset D_0, D_1 \cap M_1 = \varnothing \\
        r_1 < 1
    \end{gather*}
    Теперь мы то же соображение применим к множеству $M_2$, которое тоже нигде не плотно
    \begin{gather*}
        \exists D_2 = D_{r_2}(x_2) \subset D_1, D_2 \cap M_2 = \varnothing \\
        r_2 < \frac{1}{2}
    \end{gather*}
    и так далее $\begin{cases}
         \seq{D_n}^\infty_{n=1}, D_n = D_{r_n} (x_n), D_{n+1} \subset D_n \\
         D_n \cap M_n = \varnothing, r_n < \frac{1}{n} \end{cases}$ по теореме о вложенных шарах
         $\Rightarrow \, \exists x \in \cap^\infty_{n=1}D_n, (x \in D_n \, \wedge \, x \in X \setminus E) \Rightarrow x \notin M_n \, \forall n \Rightarrow x \notin E$
\end{proof}


\section{Полные семейства элементов}

Теперь мы будем понимать полноту в совершенно другом смысле.
Сначала вспомним, что такое линейная оболочка пространства.
\begin{definition}[Линейная оболочка]
    $X$~--- линейное пространство над $\bR (\bC)$. Рассмотрим семейство $\seq{x_\alpha}_{\alpha \in A}$~--- семейство элементов, $x_\alpha \in X$.
    \[ \calL \seq{x_\alpha}_{\alpha \in A}  = \left\{ \sum^n_{k=1} c_k x_{\alpha_k}, c_k \in \bR (\bC), n \in \bN \right\} \]
\end{definition}


\begin{definition}[Полное семейство]
    $(X, || \cdot ||)$, $\seq{x_\alpha}_{\alpha \in A}$~--- \textbf{полное семейство}, если $\overline{\calL \seq{x_\alpha}_{\alpha \in A}} = X$.
    То есть линейная оболочка всюду плотна в $X$.
\end{definition}

\begin{example}
    $C[a,b], \, \seq{x^n}^{+\infty}_{n=0}$~--- полное семейство в $C[a,b]$, так как
    $\Rho = \calL \seq{x^n}^{+\infty}_{n=0}, \, \overline{\Rho} = C[a,b]$
\end{example}

\begin{example}
    $l^p, \, 1 \leq p < + \infty, \, C_0$
    \begin{gather*}
        e_n = (1, 0, 0, \ldots, 0, \underbrace{1}_n, 0, \ldots), \seq{e_n}^\infty_{n=1} \text{~--- полное семейство} \\
        \calL \seq{e_n}^\infty_{n=1} = F \text{~--- финитная последовательность}
    \end{gather*}
\end{example}

Упражнение: $C$~--- что будет полным семейством?

\begin{statement}
    $(X, || \cdot ||)$ - нормированное пространство. В нём существует $\seq{x_n}^\infty_{n=1}$~--- полное семейство 
    \[ X \text{~--- сепарабельное} \]
\end{statement}
\begin{proof}
    Рассмотрим линейную оболочку $L = \calL \seq{x_n}^\infty_{n=1} = \seq{x = \sum^n_{j=1} c_j x_j, c_j \in \bR (\bC)}$. $\overline{L} = X$.
    \begin{gather*}
        E = \{ x = \sum^n_{j=1} c_j x_j, c_j \in \bQ \} \text{~--- счётное всюду плотное} \\
        (L \subset \overline{E} \, \wedge \, \overline{L} = X) \Rightarrow \overline{E} = X
    \end{gather*}
\end{proof}

\begin{remark}
    $l^\infty, E = \seq{ x = \seq{x_n}^\infty_{n=1}, x_n \in \bQ, \sup_{n \in \bN} |x_n| < + \infty}$.
    $\overline{E} = l^\infty$, $E$~--- не счётное.
\end{remark}
    
\section{Полные и плотные множества в $L^p$}
Сначала небольшое замечание. $(X, U, \mu)$~--- пространство с мерой $e \in U$~--- измеримые множества, 
$\chi_e(x) = \begin{cases}
    1, x \in E \\
    0, x \notin E
\end{cases}$~--- характеристическая функция. $\chi \in L^\infty(X, \mu), \, \forall e \in U$
\[ \chi_e \in L^p(X, \mu) \text{ при } 1 \leq p < + \infty \Leftrightarrow \int_X (\chi_e(x))^p d\mu < + \infty \Leftrightarrow \mu e < + \infty \]

\begin{theorem}
    $(X,U, \mu )$~--- пространство с мерой $\Rightarrow$ 
    \begin{gather*}
        \seq{\chi_e}_{e \in U} \text {~--- полное семейство в } L^\infty(X, \mu) \\
        \seq{\chi_e}_{e \in U, \mu E < + \infty} \text{~--- полное семйество в } L^p(X, \mu), 1 \leq p < + \infty 
    \end{gather*}
\end{theorem}

Для доказательства этой теоремы нужно будет вспомнить теорему Лебега из анализа (она у нас уже была).
\begin{theorem}[Лебег]
    $\seq{h_n(x)}^\infty_{n=1}$~--- измеримые, $\varphi(x)$. $\int_X \varphi(x) d \mu < + \infty, |h_n(x)| \leq \varphi(x)$ п.в. на $X$
    \[ h_n(x) \underset{n \to \infty \text{ п.в. по } \mu}{\longrightarrow} h(x) \Rightarrow \liml_{n \to \infty} \int_X h_n(x) d \mu = \int_X h(x) d \mu \]
    
\end{theorem}

\begin{proof}
    Вспомним конструкцию, которая была в математическом анализе. $f$~--- измеримая, $f(x) \geq 0, x \in X$.
    Рассмотрим разбиение множества $X$, а по нему построим соотвествующую простую функцию
    \begin{gather*}
        n \in \bN \quad e_k = \seq{x \in X: \frac{k}{n} \leq f(x) < \frac{k+1}{n}}, k = 0, 1, \ldots, n^2 - 1 \\
        e_{n^2} - \seq{x: f(x) \geq n} \Rightarrow X = \bigcup^{n^2}_{k=0} e_k, e_k \cap e_j = \varnothing (k \ne j)
    \end{gather*}

    Теперь построим измеримые функции, потом они будут простыми.
    \[g_n(x) = \sum^{n^2}_{k=1} \frac{k}{n} \chi_{e_k} \quad 0 \leq g_n(x) \leq f(x), x \in X \]
    $f(x) \leq g_n(x) + \frac{1}{n}, x \in \bigcup^{n^2-1}_{k=0} e_k$ \\

    Теперь все готово, чтобы обсудить случай $L^\infty$. Пусть $f \in L^\infty(X,\mu) \Rightarrow n \geq ||f||_\infty \Rightarrow \mu(e_{n^2}) = 0 $.
    $\Rightarrow |f(x) - g_n(x)| \leq \frac{1}{n}$ для п.в. $x \in X$ \\
    $\Rightarrow ||f-g_n||_\infty \underset{n \to \infty}{\longrightarrow} 0, g_n \in \calL \seq{\chi_e}_{e \in U}$

    $\Rightarrow f \in \overline{\calL \seq{\chi_e}_{e \in U}}$ \\
    Посмотрим теперь, что происходит с конечными $p$. Тут вспоминаем теорему Лебега, она была верна для интеграла Лебега, но верна и для произвольной меры.
    \begin{gather*}
        \begin{cases}
            f(x) \in L^p(X, \mu), 1 \leq p < + \infty \quad |f(x) - g_n(x)|^p \leq |f(x)|^p \\
            g_n(x) \underset{\text{п.в.}}{f(x)} \quad \Rightarrow |f(x) - g_n(x)|^p \underset{n \to \infty}{\longrightarrow} 0
        \end{cases} \stackrel{\text{Лебег}}{\Rightarrow} \\
        \text{ все, что надо~--- убедиться, что мера конечная}
        \liml_{n \to \infty} \left( \int_X |f-g_n|^p d \mu \right)^\frac{1}{p} = 0 \\
        f \in L^p \Rightarrow \mu e_k < + \infty \quad f(x) \geq \frac{k}{n}, x \in e_k \Rightarrow \left( \int_X |f|^p d\mu \right)^\frac{1}{p} \geq \left( \int_{e_k} \left( \frac{k}{n} \right)^p d\mu \right)^\frac{1}{p} = \\
        \frac{k}{n} (\mu e_k)^\frac{1}{p} \Rightarrow \mu e_k < + \infty \\
        \Rightarrow f \in \overline{\calL \seq{\chi_e}_{e \in U, \mu e < + \infty}}
    \end{gather*}
    Теперь покажем, что для произвольных $f$ рассуждение тоже верно.  Рассмотрим замыкание линейное оболчоки
    \begin{gather*}
        \begin{cases}
            f: X \rightarrow \bR, \Rightarrow f = f_+ - f_-, f_+(x) \geq 0, f_-(x) \geq 0 \\
            f: X \rightarrow \bC \Rightarrow f = u + iv; u, v: X \rightarrow \bR
        \end{cases} \Rightarrow \\
        \forall f \in L^p, f \in \overline{\calL \seq{\chi_e}_{e \in U}} \\
        (p = \infty \, \forall e, p < + \infty, \mu e < + \infty)
    \end{gather*}
\end{proof}

Теперь, зная эту теорему, посмотрим, какое множество будет полным в пространстве $l^\infty$

\begin{corollary}
    $l^\infty, A \subset \bN$, $X^A = \seq{x^A_n}^\infty_{n=1}, X^A_n = \begin{cases}
        1, n \in A \\
        0, n \notin A
    \end{cases} \Rightarrow$
    \[ \seq{X^A}_{A \subset \bN} \text{~--- полное семейство в } l^\infty \]
\end{corollary}

\begin{proof}
    $l^\infty = L^\infty(\bN, \mu), \mu(n) = 1 \, \forall n \in \bN \quad \forall A \subset \bN, A $~--- измеримо 
    \[\chi_A = X^A \Rightarrow \seq{X^A}_{A \subset \bN} \text{~--- полное семейство } \] 
\end{proof}


\begin{theorem}
    $(\bR^n, U, \lambda)$, $\lambda$~--- классическая мера Лебега. $U$~--- измеримые по Лебегу множества. 
    \[ \R = \left\{ \Delta = \prod^n_{j=1} [a_j, b_j), a_j < b_j; a_j, b_j \in \bR \right\} \text{~--- множество ячеек} \]
    \[\Rightarrow \seq{\chi_\Delta}_{\Delta \in \R} \text{~--- полное семейство в } L^p(\bR^n, \lambda), 1 \leq p < + \infty \]
\end{theorem}
Достаточно рассмотреть характеристические множества ячеек.

\begin{proof}
    Собираемся приблизить множество линейной комбинацеий характеристических функций ячеек. Вспомним определение внешней меры.
    \begin{gather*}
        e \in U, \lambda(e) < + \infty \\
        \lambda(e) = \inf \left\{ \sum^\infty_{k=1} \lambda(\Delta_k), e \subset \bigcup^\infty_{k=1} \Delta_k, \Delta_k \in \R, \Delta_k \cap \Delta_j = \varnothing \right\}
    \end{gather*}
    Сначала просто по определению нижней грани. $\forall \varepsilon > 0 \Rightarrow \, \exists \seq{\Delta_k}^n_{k=1}$.
    $\lambda(e) \leq \sum^\infty_{k=1} \lambda(\Delta_k) < \lambda(e) + \varepsilon$. \\
    $e \subset \cup^\infty_{k=1} \Delta_k, \Delta_k \in \R, \Delta_k \cap \Delta_j = \varnothing$ при $k \ne j$. \\
    \begin{gather*}
        A = \bigcup^\infty_{k=1} \Delta_k, e \subset A, \lambda(A \setminus e) < \varepsilon \\
        \exists N \in \bN \quad \sum^\infty_{k=N+1} \lambda(\Delta_k) < \varepsilon, B = \bigcup^n_{k=1} \Delta_k \\
        \Rightarrow \lambda(A \setminus B) < \varepsilon 
    \end{gather*}
    \begin{multline*}
        || \chi_e - \chi_b ||_p \leq ||\chi_e - \chi||_p - ||\chi_A - \chi_B||_p \leq \\
        \left( \int_{A \setminus e} \mathbb{1} d\mu \right)^{\frac{1}{p}} + \left( \int_{A \setminus B} \mathbb{1} d\mu \right)^{\frac{1}{p}} < \varepsilon^{\frac{1}{p}} + \varepsilon^{\frac{1}{p}} = 2 \varepsilon^{\frac{1}{p}} 
    \end{multline*}

    \begin{gather*}
        \chi_b = \sum^N_{k=1} \chi_{\Delta_k} \in \calL \seq{\chi_\Delta}_{\Delta \in \R} \\
        \begin{cases}
            \overline{\calL \seq{\chi_e}_{e \in U}} = L^p \\
            \chi_e \in \overline{\calL \seq{\chi_\Delta}_{\Delta \in \R}}
        \end{cases} \Rightarrow \overline{\calL \seq{\chi_\Delta}_{\Delta \in \R}} = L^p, 1 \leq p < + \infty
    \end{gather*}

\end{proof}

\begin{corollary}
    $E \subset \bR^n$, $E$~--- измеримые по Лебегу, $1 \leq p < + \infty$ 
    \[ \Rightarrow L^p(E, \lambda) \text{~--- сепарабельные} \]
    ($\lambda$~--- мера лебега)
\end{corollary}

\begin{proof}
    Докажем, что $L^p(\bR^n, \lambda)$~--- сепарабельное.
    \[ \R = \left\{ \Delta = \prod^n_{j=1} [a_j, b_j), a_j < b_j, \, a_j, b_j \in \bR \right\} \text{~--- полные семейства в } L^p \]
    Теперь мы возьмём только такие ячейки, полуинтервалы которых мы перемножаем, имеют рациональные концы. Пока что можем
    сказать, что это счётное множество.
    \[  R_0= \left\{ \Delta = \prod^n_{j=1} [a_j, b_j), a_j < b_j, \, a_j, b_j \in \bQ \right\} \text{~--- счётное множество} \]

    \begin{gather*}
        \Delta \in \R \quad \let \varepsilon > 0 \\
        \Rightarrow \, \exists \Delta_0 \in R_0, \Delta \subset \Delta_0, \lambda(\Delta_0 \setminus \Delta) < \varepsilon \\
        \Rightarrow || \chi_{\Delta_0} - \chi_{\Delta} ||_p = || \chi_{\Delta_0 \setminus \Delta} ||_p = \left( \int_{\Delta_0 \setminus \Delta}\mathbb{1} dx \right)^{\frac{1}{p}} = 
        \left( \lambda (\Delta_0 \setminus \Delta)\right)^{\frac{1}{p}} < \varepsilon^{\frac{1}{p}} \\
        \Rightarrow \, \forall \Delta \in \R \, \chi_\Delta \in \overline{\calL \seq{\chi_\Delta}_{\Delta \in R_0}}
     \end{gather*}

     $R_0$~--- полное счётное семейство $\stackrel{\text{утверждение}}{\Rightarrow} L^p(\bR^n, \lambda)$~--- сепарабельное.
     \begin{gather*}
        E \subset \bR^n, E \text{~--- измеримое }, f \in L^p(E, \lambda) \\
        \text{ пусть } f(x) = 0, x \in \bR^n \setminus E \Rightarrow f \in L^p(\bR^n, \lambda) \\
        \Rightarrow L^p(E, \lambda) \text{~--- подпространство } L^p(\bR^n, \lambda) \Rightarrow L^p(E, \lambda) \text{~--- сепарабельно}
     \end{gather*}
\end{proof}


\begin{definition}
    $(X, U, \mu)$~--- пространство с мерой. $(X, \rho)$~--- метрическое пространство.
    $\mu$ \textbf{~--- борелевская мера}, если $(G$~--- открытое $\Rightarrow G \in U)$
\end{definition}

$\beta$~--- минимальная $\sigma$-алгебра,  содержащая все открытые множества. $\beta$~--- \textbf{борелевские множества}, то есть $\beta \subset U$.

Чем же хороши борелевские меры? Оказывается, они безумно связаны с непрерывными функциями

\begin{remark}
    Пусть $f: X \rightarrow \bR, f$~--- непрерывная $\Rightarrow f^{-1}((c,+\infty)), c \in \bR, (c, _\infty)$~--- открытое в $\bR$.
    Определение непрерывной функции из топологии: прообраз любого открытого множества открыт.
    Так как прообраз $f$ открыт в $X$ $\Rightarrow f$~--- измеримая по $\mu$, если $\mu$~--- борелевская.

\end{remark}

\begin{remark}
    $\lambda$~--- мера Лебега в $\bR^n$, тогда $\lambda$~--- борелевская.
\end{remark}
Еще более специальное определение. Этим свойством мера Лебега тоже обладает.

\begin{definition}[регулярная мера]
    $(X, U, \mu)$, $(X, \rho)$, $\mu$~--- борелевская. $\mu$~--- \textbf{ регулярная мера}, если $\forall e \in U $
    \[ \sup_{\seq{F \subset e, F \text{~--- замкнутое} }} \seq{ \mu(F)} = \mu e = \inf_{\seq{e \subset G, G \text {~--- открытое}}} \mu G \]
\end{definition}

\begin{remark}
    $\lambda$-мера Лебега~--- регулярная.
\end{remark}

На самом деле эти 2 свойство друг из друга следуют, но мы это доказывать не будем.

\begin{theorem}
    $(X, U, \mu)$, $(X, \rho)$, $\mu$~--- регулярная мера $\Rightarrow$  непрерывная функция плотна В
    $L^p(X, \mu), 1 \leq p < + \infty$.
    \[ \overline{C(X) \cap L^p(X\mu)}^{|| \cdot ||_p} = L^p(X, \mu) \]
\end{theorem}
\begin{proof}
    Мы уже знаем, что полное семейство~--- это семейство характеристических функций всех
    измеримых функций, и мы будем этим изо всех сил пользоваться. Возьмём какую-то характеристическую функцию из множества и ее будем приближать
    непрерывными функциями. \\
    $\seq{ \chi_e}_{e \in U, \mu e < + \infty }$~--- полное семейство. \\
    пусть $ e \in U, \mu e < + \infty, \, \let \varepsilon > 0, \mu$ ~--- регулярная $\Rightarrow 
    \, \exists F \subset e \subset G, F$~--- замкнутое, $G$~--- открытое. $\mu(G \setminus F) < \varepsilon$
    
    %символическая картинка
    Когда мы попадем в $X \setminus G$ она будет равна нулю.
    \[ \varphi(x) = \frac{\rho(x, X \setminus G)}{\rho(x, X \setminus G) + \rho(x, F)} \]
    Нужно позаботиться о том, чтобы знаменатель не был равен нулю. \\
    $\rho(x,A)$~--- непрерывная функция $\forall A \subset X$. $X \setminus G$~--- замкнутое, $F$~--- замкнутое.
    Если $\rho(x, F) = 0 \Rightarrow x \in F \Rightarrow x \notin X \setminus G \Rightarrow \rho(x, X \setminus G) > 0$ 
    \[ \Rightarrow \rho(x, X \setminus G) + \rho(x, F) > 0 \, \forall x \in X \Rightarrow \varphi \in C(X) \]

    $\varphi(x) = 0, x \in X \setminus G, \varphi(x) = 1, x \in F \quad \forall x \in X \: 0 \leq \varphi(x) \leq 1$

    Понятно, что модуль $\varphi(x)$ совпадает с характеристической функцией множества $e$.
    \begin{gather*}
        \chi_e(x) - \varphi(x)| \leq 1 \quad \forall x \in X \\
        \chi_e(x) - \varphi(x) = 0 \quad x \in F \text{ или } x \in X \setminus G \\
        \Rightarrow ||\chi_e - \varphi||_p = \left( \int_X |\chi_e(x) - \varphi(x) |^p d\mu \right)^{\frac{1}{p}} = 
        \left( \int_{G \setminus F} |\chi_e(x) - \varphi(x) |^p d \mu \right)^{\frac{1}{p}} \leq \\
        \leq \left( \mu(G \setminus F) \right)^{\frac{1}{p}} < \varepsilon^{\frac{1}{p}} \\
        \Rightarrow \chi_e \in \overline{C(X)}^{|| \cdot||_p}
    \end{gather*}
    Тем самым мы доказали, что $\chi_e(x)$ может быть приближена непрерывными функциями.
    Может быть, стоить отметить, что 
    $\mu G < \mu e + \varepsilon < +\infty \quad \int_X |\varphi(x)|^p d \mu - \int_G |\varphi(x)|^p d \mu < \mu G \Rightarrow
    \varphi \in L^p(X, \mu)$
    
\end{proof}


Раз утверждение верно для любых регулярных мер, то оно верно и для меры Лебега.

\end{document}
