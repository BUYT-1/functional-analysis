\declaretheoremstyle[
spaceabove=12pt, spacebelow=12pt, headindent=6pt,
headpunct=,
headfont=\normalfont\bfseries,
notefont=\mdseries, notebraces={(}{)},
bodyfont=\normalfont,
postheadspace=1em,
qed=\qedsymbol
]{proofStyle}

\declaretheoremstyle[
spaceabove=6pt, spacebelow=6pt,
headfont=\normalfont\bfseries,
notefont=\mdseries, notebraces={(}{)},
bodyfont=\normalfont,
postheadspace=1em,
]{theoremStyle}



\tcbset{sharp corners=all, colback=white}
\tcolorboxenvironment{theorem}{}
\tcolorboxenvironment{theorem*}{}
\tcolorboxenvironment{lemma}{}
\tcolorboxenvironment{proposition}{}
\tcolorboxenvironment{corollary}{}
\tcolorboxenvironment{definition}{}
\tcolorboxenvironment{property}{}

\declaretheorem[name=Теорема, numberwithin=chapter, style=theoremStyle]{theorem}
\declaretheorem[name=Теорема, numbered=no, style=theoremStyle]{theorem*}
\declaretheorem[name=Свойство, numberwithin=chapter, style=theoremStyle]{property}
\declaretheorem[name=Свойство, numbered=no, style=theoremStyle]{property*}
\declaretheorem[name=Лемма, numberwithin=chapter, style=theoremStyle]{lemma}
\declaretheorem[name=Лемма, numbered=no, style=theoremStyle]{lemma*}
\declaretheorem[name=Следствие, numberwithin=chapter, style=theoremStyle]{corollary}
\declaretheorem[name=Следствие, numbered=no, style=theoremStyle]{corollary*}
\declaretheorem[name=Пример, numberwithin=chapter, style=theoremStyle]{example}
\declaretheorem[name=Пример, numbered=no, style=theoremStyle]{example*}
\declaretheorem[name=Определение, numberwithin=chapter, style=theoremStyle]{definition}
\declaretheorem[name=Определение, numbered=no, style=theoremStyle]{definition*}
\declaretheorem[name=Замечание, numberwithin=chapter, style=theoremStyle]{remark}
\declaretheorem[name=Замечание, numbered=no, style=theoremStyle]{remark*}
\declaretheorem[name=Доказательство, numbered=no, style=proofStyle]{Доказательство}